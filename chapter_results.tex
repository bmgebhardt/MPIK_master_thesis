% !!!!!!!!!!!!!!!!!!!!!!!!!!!!!!!!! !
% before compiling run:
% >>> export TEXINPUTS=/home/gebhardt/ownCloud/00_WriteUP/03_Thesis_Proposal/TU_Da_Layout
% for the missing TUDa Layout packages
% note: fonts still missing not anymore!!!!!!!!!!@!R4W323 http://tudadesign.github.io/installation_miktex.html
% !!!!!!!!!!!!!!!!!!!!!!!!!!!!!!!!!
\documentclass[12pt,article,type=msc,colorback,accentcolor=tud9c]{tudthesis}
%\usepackage{ngerman}
%\usepackage{ngerman}
\usepackage{graphicx}
\usepackage{caption}
\usepackage{wrapfig}
\usepackage{amsmath}
\usepackage{enumitem}
%\usepackage{hyperref}
\usepackage[percent]{overpic}
\usepackage{url}
\usepackage{afterpage}
\usepackage{lineno}
\linenumbers

\renewcommand{\baselinestretch}{1.2} 

\newcommand{\getmydate}{%
\ifcase\month%
\or Januar\or Februar\or M\"arz%
\or April\or Mai\or Juni\or Juli%
\or August\or September\or Oktober%
\or November\or Dezember%
\fi\ \number\year%
}

\newcommand\blankpage{%
    \null
    \thispagestyle{empty}%
    \addtocounter{page}{-1}%
    \newpage}



\newcommand*\rfrac[2]{{}^{#1}\!/_{#2}}
%\usefont{T1}{ptm}{b}{it}
\begin{document}
%\usefont{T1}{ptm}{b}{it}
\thesistitle{Silicon Photomultiplier Classification of the GCT Camera of CTA} %
{Silicon Photomultiplier Klassifikation der GCT Kamera von CTA}
\author{Ben Gebhardt}
\birthplace{Heidelberg}
\referee{Dr. Richard White (MPIK)}{Prof. Jim Hinton (MPIK)}[Prof. Tetyana Galatyuk (TU DA)]
\department{Fachbereich Physik}
\group{Max Planck Institut f"ur Kernphysik Heidelberg}
\dateofexam{\today}{\today}
%\tuprints{12345}{1234}
\makethesistitle


\clearpage
\begin{abstract}
whats this about
\end{abstract}
\afterpage{\blankpage}



\clearpage
\begin{abstract}
worum es geht
\end{abstract}
\afterpage{\blankpage}



\clearpage
\tableofcontents
\clearpage
%\newpage
%______________________________________________________________________________________________________________________________________________________________________________________________________________
\section{\Large Cosmic Radiation}
%
\begin{figure}[b]
\begin{centering}
%L, B, R, T
\resizebox{0.9\columnwidth}{!}{\includegraphics[trim=0cm 0cm 0cm 0, clip=true]{D:/OwnCloudData/00_WriteUP/04_Thesis/Pic/Proposal/Fig/{01_astroparticle_field}.jpg}}
\caption{Gamma radiation photons (yellow) and scattered cosmic ray protons (blue) from an astrophysical source arriving on Earth. Neutrinos (grey) mostly do not interact. Picture from \cite{ungCTA}}
\label{fig:GammaRad}
\end{centering}
\end{figure}%
\noindent
Cosmic rays consist of a single particle with energies from 10$^{10}$ to 10$^{20}$~eV and describes charged particles hitting the Earth's atmosphere. They where discovered by V.F. Hess in 1912 during the famous baloon flight experiments. He aimed to measure the conductivity of air, that until then, was believed to be an insulator resulting in some problems regarding the discharge of an electrically charged body, no matter how well it was isolated from the ground. Hess found the airs conductivity to increase with higher altitude, concluding the presence of a large amount of ionizing radiation above the atmosphere. The most energetic laboratory based accelerators operate in the 10$^{12}$~eV energy range.\\
Cosmic rays do not include those low energy particles originating from our sun. With a particle energy up to 1keV, those are referred to as solar wind, which means, by definition, cosmic rays arrive on Earth from outside our solar system. They consist of 87$\%$ protons, 12$\%$ $\alpha$-particles , 1$\%$ heavier nuclei and some electrons. High energy comsic rays hitting Earth are very rare, averaging to one per year in an area of one square kilometer. Except at those very high energies (>10$^{18}$eV) cosmic rays will not reach earth directly and can not be observed to pinpoint their source. Traveling through the interstellar medium , they get scattered by the interstellar magnetic fields, the cosmic microwave background and other hindrances and therefore have lost all directional information. However, directly observable cosmic rays, for example at the Pierre-Auger Observatory, provide an insight into cosmic particle accelerators.\cite{CTADesign}\\
Cosmic rays of the higher energies are therefore observed via a detour: Gamma radiation. \\



%______________________________________________________________________________________________________________________________________________________________________________________________________________
\begin{wrapfigure}{R}{0.5\textwidth}
\includegraphics[width=0.5\textwidth]{D:/OwnCloudData/00_WriteUP/04_Thesis/Pic/Proposal/Fig/{Fermi_image01}.jpg}
\caption{\label{fig:LAT}FermiLAT Picuture from \cite{FermiLAT}}
\end{wrapfigure}
Gamma radiation cannot be generated by thermal emission of hot stelar objects, the only event with a high enough temperature to produce thermal radiation in the range of GeV and TeV gamma radiation would be the big bang, there is and has been nothing else in the known universe. If thermal radiation reflects the temperature of the emitting body, what do gamma rays tell us?\\
Gamma radiation probe a non-thermal universe. In this you need other mechanisms to concentrate large amounts of energy into a single quantum.\\
There are many diverse mechanisms of emitting gamma radiation. Gamma rays are generated by high relativistic particles, in a first step for example: accelerated by the shockwave of a supernova explosion. Those cosmic rays then collide with ambient gas, interact with photons or magnetic fields, by inverse compton scattering, emitting high energy photons (down-top). Very-high-energy (VHE) gamma radiation is defined in the energy range of 10$^{11}$ to 10$^{14} $eV.

One such source of VHE gamma radiation and also the most famous, because the first to be discovered, lies within the Crab Nebula. The Neutron Star located inside the Crab Nebula is a Pulsar and the remnant of Supernova1054 and steadily emits gamma radiation energies up to 80TeV. Another compound of the gamma radiation here is the so called Pulsar Wind Nebula. It is composed of highly relativistic charged particles from the Pulsars giant rotating magnetic field interacting with the expanding Supernova remnant via inverse compton scattering. Supernova shockwaves themselfes can also drive atomic nuclei to high energies which in turn generate observable gamma-rays in a top-down fashion. Additionaly, binary systems consisting of a black hole or pulsar orbiting a massive star can emit a flow of high-energy particles with variing intensity, based on the elliptical orbit, where particle acceleration conditions vary.\\
So just like thermal radiation reflects the temperature of the emitting object, the flux and energy spectrum of the gamma rays reflect the flux and spectrum of the high energy particles. So they can be used to trace these comsic rays and electrons in distant regions of our own galaxy or even beyond. \\
One surprise was the discovery of so called "dark sources" , objects emitting VHE gamma-rays, but have no counterpart in other wavelenghts, meaning those objects might only be observable through gamma-rays. In extragalactical regions, gamma rays provide information on active galaxies, where a constant stream of gas feeds a supermassive black hole at the center, releasing enourmus amounts of energy. From there, gamma rays are believed to be emitted, giving insight into one of the most violent but to date poorly understood environments in our universe. 
Even higher energy gamma rays could also be the product of decays of heavy particles, like dark matter or cosmic strings. They therefore also provide a window to the discovery of dark matter.\\
Gamma Radiation carries unique information about the most energetic phenomena in our universe. 
The only problem is, our atmosphere is opaque for gamma radiation, gamma ray astronomy in the lower energies is done by satellite based instruments like FermiLAT. The Large Area Telescope, the principle scientific instrument on the Fermi Gamma-Ray Space Telescope Spacecraft sensitive in the energy range between 20 MeV and 100 GeV, launched in June 2008. fig (\ref{fig:LAT}) To reach the higher energy range through space based telescopes is very inconvinient, since the required mass the telescope active area would need to detect the gamma rays increases with energy, and can therefore by very expensive to launch into space.







\subsection{Air shower induced Cherenkov Radiation}
% farmiLAT\\
% particle shower\\
% triangulation of the stereoscopic view\\

%% FIG Cherenkov
\begin{wrapfigure}{R}{0.4\textwidth}
\centering
\includegraphics[width=0.4\textwidth]{D:/OwnCloudData/00_WriteUP/04_Thesis/Pic/Proposal/Fig/{gamma}.jpg}
\caption{\label{fig:Cherenkov} The cone of Cherenkov light emitted by an extensive air shower. Picture from \cite{AsperaCTA}}
\end{wrapfigure}

%shower physics
Because our atmosphere is opaque for gamma radiation, gamma ray astronomy was mainly done by satellite based instruments like FermiLAT.
That said, we can still see their effects on earth through gamma-ray induced particle cascades. When a primary particle, i.e. a gamma photon or cosmic ray enters the atmosphere and collides with a nucleus of the air, it gets scattered and creates secondary electrons, positrons and photons. Those secondary particles also interact with the atmosphere creating a cascade of particles called a particle shower. 




In this air shower, the initial and each subsequent particle traveling through our atmosphere emitts Cherenkov light. Cherenkov radiation is a phenomenon caused by charged particles traveling faster than the local speed of light would allow in that medium. This light is emitted in a narrow cone with an increasing angle as the particles travel downward. This Cherenkov light shows as a very short ($\sim$5ns) flash with a peak in the UV-spectrum at around 330nm.
Thus, we can image the particle cascade measured with the telescope and can reconstruct their path through a stereoscopic image of the shower taken by multiple telescopes(appendix {\ref{app:CTAPATH}}), reconstructing the position of the source in the sky.\\

Is is also possible to reconstruct the energy of the original photon from the amount of light produced. This is possible because energy is conserved, so all energy of the original photon is now distributed between the particles of the shower.\\

To determine whether it is a hadronic shower originating from cosmic rays or a gamma shower, originating from gamma ray photons, we look at many different shower characteristics like diffusion , compactness, and the ratio of width and length.
\\\\
notes:
 Cosmic Ray origin ? , energy spectrum? knee, ankle, power law?
 missing? necessary? Shower schematic overview and propagation





%______________________________________________________________________________________________________________________________________________________________________________________________________________


\clearpage
%\newpage
\section{\Large Imaging Atmospheric Cherenkov Telescopes}




%------------------------------Telescopes


Before, gamma ray astronomy was mainly done using satelite based instruments. The technique, pioneered by the Wipple Collaboration (US), behind the ground based experiments called Imaging Atmospheric Cherenkov Telescopes (IACTs) aims at measuring the time, direction and energy of flashes of Cherenkov light from extensive air showers caused by very-high-energy (VHE) gamma radiation. Those ground based instruments have a much larger effective detection area than any satelite based instrument, which have a typical detection size of 1m$^2$. The range of the Cherenkov flash being between 300-600nm, current generation Silicon Photomultipliers (SiPMs) are a promising candidate to replace the progenitor photon detector used in previous experiments like HESS, the Photomultiplier Tube (PMT).\\
Current ground based IACTs consisted of mostly 4 telescopes,  HESS, MAGIC, Veritas among them. (appendix{\ref{app:IACTProjects}})\\
All of those arrays consist of at most 5 telescopes spread over a wide area. So most cascades are viewed by only 2 or 3 of the telescopes. Additionaly, due to the low flux of VHE gamma radiation, detectors for this energy range are spread over a large area, making space based instruments, which detect the incident gamma ray, an inconvenient choice. Another detection concept for VHE gamma radiation are ground based air shower particle detectors such as the High-Altitude-Water-Cherenkov observatory (HAWC) \cite{HAWC}. It employs a similar detection principle, recording shower particles reaching arrays of ground based particle detectors filled with water as detection medium, in contrast to air in IACTs. Those have the advantage of a larger duty cycle than IACTs, as they are able to operate during the day. Their limited sensitivity even with high observation time however will not allow them to compete with the sensitivity and resolution of IACTs such as in the CTA. The array will however be able to provide useful complementary information. 

\subsection{Cherencov Telescope Array}



The Cherenkov Telescope Array, CTA, is an advanced ground-based observatory array of many tens of telescopes distributed over a larger energy range than before. It will allow detection of gamma rays over a large area on the ground and from multiple different directions. The array will consist of 60 - 100 telescopes of different designs and sizes to cover the aimed for energy range and area. Science goals are the understanding of cosmic rays and their role in the universe, including the study of cosmic particle accelerators, such as pulsars, pulsar wind nebulae , supernova remnants and gamma ray binaries. Secondly particle acceleration around black holes of supermassive or stellar size and lastly physics beyond the Standard Model. 
There are currently 2 sites planed, which when deployed, will achieve full-sky coverage.
\begin{enumerate}
\item A southern cta site, with an array consisting of three types of telescopes:\\
\begin{enumerate}
\item LST\footnote{LST large scale telescope} The low energy instruments, between 20 and 200 GeV, will consist of four 23 meter class telescopes with a moderate field of view (FoV) of the order of about 4.5 degrees.
\item MST\footnote{MST medium size telescope} The medium energy range, from around 100 GeV to 10 TeV, will be covered by 24 telescopes of the 12 meter class with a FoV of 7 degrees.
\item SST\footnote{SST small scale telescope} The high energy instruments , operating between a few TeV to 300 TeV, will consist of about 72 small (4 meter diameter) telescopes with a FoV ranging from 9.1 to 9.6 degrees.
\item  (area covered by the array of telescopes: ~4 km$^2$)\\
\end{enumerate}
\item CTA Northern Site:
\begin{enumerate} 
\item 4 large-size telescopes and 15 medium-size telescopes
\item (area covered by the array of telescopes: ~0.4 km$^2$)
\end{enumerate}
\end{enumerate}

There are currently 3 different prototypes in development for the SST variant: SST2M GCT and Astri , and a single mirror prototype SST1M\\
This papers work is conducted on the GCT design, one of the prototypes for the high energy telescope called SST-2M GCT utilizing a 2 mirror design called Schwarzschild-Couder.

\subsection{GCT}

\begin{figure}[t]
\begin{centering}
%L, B, R, T
\resizebox{0.5\columnwidth}{!}{\includegraphics[trim=0cm 0cm 0cm 0, clip=true]{D:/OwnCloudData/00_WriteUP/04_Thesis/Pic/Proposal/Fig/{Bild1}.png}}
\caption{If possible replace picture with GCT with mounted CHEC camera picture here Picture from }
\label{fig:GCT_Pic}
\end{centering}
\end{figure}


The 2 mirror design of the telescope allows us to utilize SiPMs. Current IACTs have a parabolic optical system, which is reliable and efficient but they need to be large in order to have a large FoV due to aberrations, resulting in huge CAMERA plate scales and therefore expensive assembly of photodetectors.\\

Schwarzschild-Couder have no such aberrations, at least not on our scale, while the IACTs can still have a very large Field of View (FOV) up to $15\,^{\circ}$ without significant degradation of the spot size. The resulting physical pixel size is more compact than that of the single mirror optics and cost-effective  photon sensors such as multi-anode photomultipliertubes (MAPMTs) (like in CHEC-M) or Silicon Photomultipliers (SiPMs) can be used for the camera.\\
There are of course disadvantages, the optical system will be more complex, with tighter tolerances.\\

In summary it reduces the camera plate scale, allowing us to use SiPM photosensors.

\subsection{CHEC-S}
The Compact High Energy Camera, or CHEC is one of three prototype camera concepts in development for one of the SST structures within CTA. One camera features 2048 SiPMs, building equaly many readout channels per camera. The readout is done by so called TARGET\footnote{TARGET TeV Array Readout with $\frac{GS}{s}$ sampling and Event Trigger} modules for sampling, digitization and triggering, one of those consists of 4 ASICs\footnote{ASIC Application Specific Integrated Circuit} with 16 channels each. The camera houses 32 TARGET modules, one of them responsible for readout of 64 SiPMs. The TARGET modules build the front-end of the integrated electronics inside the CHEC-S camera connecting the SiPM buffer to the cameras backplane. After the SiPM is triggered the buffer amplifies the signal, that is then send to the ASIC inside the TARGET modulde, where an analog-to-digital converter and a shaper convert the signal before it is send to the backplane for transfer to the main array hub. 

\subsection{SiPM requirements for CTA}
\begin{figure}[t]
\begin{centering}
\resizebox{0.7\columnwidth}{!}{\includegraphics[trim=0cm 0cm 0cm 0cm, clip=true]{D:/OwnCloudData/00_WriteUP/04_Thesis/Pic/Intro/Cherenkov_Spectrum.JPG}}
\caption{The spectrum of Cherenkov light observed from an extenden air shower at 2200m asl. compared to the expected NSB measured in La Palma. Units on the y-axis are arbitrary because NSB and Cherenkov light vary by different parameters. Image from \cite{SiPMvsMAPMT}}
\label{fig:Cherenkov_NSB}
\end{centering}
\end{figure}

In order to make an educated choice on the best SiPM device to use to populate the cameras focal plane, an in-depth characteristic study on pulse-shape, gain, temperature-dependence, detection efficiency, thermal noise and correlated secondary effects is conducted by multiple groups within the CTA collaboration.\\
This paper studies noise from thermal, as well as correlated secondary effects and their temperature dependence. 
To be considered a promising candidate, any given SiPM device must have a high enough fill-factor (detector space versus dead space) to guarantee high Photon Detection Efficiency (PDE). This is necessary to ensure the data loss in CTA requirements? . The peak of the spectral response of the SiPM is desired to be around the spectral peak of the Cherenkov light and at the same time, must have a fast enough drop to ensure less Night Sky Background (NSB) pick up (Figure(\ref{Cherenkov_NSB})). The overall noise from thermal and correlated secondary effects of the SiPM must be sufficiently below the expected NSB rate at the location (usually around $\sim$20-80 MHz).\\\\
 Additional?



%______________________________________________________________________________________________________________________________________________________________________________________________________________
\clearpage
%\newpage
\section{\Large Silicon Photomultipliers}
\label{sec:SiPM}

\begin{figure}[h]
\begin{centering}
%L, B, R, T
\resizebox{0.6\columnwidth}{!}{\includegraphics[trim=0cm 0cm 0cm 0, clip=true]{D:/OwnCloudData/00_WriteUp/04_Thesis/Pic/SiPM_Physics/{SiPM_PMT.JPG}}}
\caption{The size of the photodetector still used in progenitor IACT experiments (PMT)(left) compared to an SiPM(right). Picture from \cite{RWTHMaster1}}
\label{fig:PMT_SiPM_Size}
\end{centering}
\end{figure}

Silicon Photomultipliers (SiPMs) are semiconductor photo detectors, that have attracted increased attention over the last decade for their possible use in astroparticle physics. The sensor consists of an array of avalanche photo-diodes, typically ~50$\mu$m in size. Depending on the pixel-size, one pixel contains several 1000 diodes, hereafter called cells. Each cell is a PN junction fig(\ref{fig:SiPM_scheme}) supplied with a reverse bias-voltage above breakdown, which is called operation in Geiger-mode, in analogy to the Geiger counter. In this mode, a photon or thermal excitation will produce an electron-hole pair in the depleted region. Through impact ionization these charge carriers will in turn trigger an avalanche in a cell, which in turn generates a large output pulse typically in the range of several Mega electronvolt. This avalanche is then passively quenched by a resistor to limit the current in the substrate and to reset the cell to a quiet state so it is photosensitive again. The signal of the SiPM is the sum of the signal of all cells, read out over their quenching resistor via a common output.\\

\begin{figure}[h]
\begin{centering}
%L, B, R, T
\resizebox{0.6\columnwidth}{!}{\includegraphics[trim=0cm 0cm 0cm 0, clip=true]{D:/OwnCloudData/00_WriteUp/04_Thesis/Pic/SiPM_Physics/{SiPM_scheme_HPK.JPG}}}
\caption{Structure and carrier multiplication through an avalanche inside a SiPM. Picture from \cite{HPK_SiPM}}
\label{fig:SiPM_scheme}
\end{centering}
\end{figure}



Silicon Photomultipliers posses major advantages over their progenitor, the Photomultipliertubes, or PMT fig(\ref{PMT_SiPM_Size}). They are more resistent to mechanical and accidental light-exposure damage through ambient light. Silicon Photomultipliers have a lower power consumption and, operating at a much lower bias-voltage, there is no need for high-voltage as in PMT. They posses a high photon detection efficiency and are insensitive to magnetic field changes. There is rapid improvement, being a fairly new technology, with new generations every $\sim$5 months and decreasing costs per mm$^2$. Viewed over all cells of the whole pixel, fluctuations in the gain are very small. This is because of the uniformity during manufacture and visible in the width and the clear resolution of the p.e. peaks in the pulse area spectrum. Examples in the appendix {\ref{app:PAS_Clean}}. Multi anode photomultipliertubes do not posses those small gain fluctuations, which is due to their structure. These make Silicon Photomultipliers an interesting candidate in astrophysics experiments for both, space- and ground-based telescopes (or IACTs, Imaging Atmospheric Cherenkov Telescopes). 
\begin{enumerate}
\item Sturdiness
\item May be exposed to ambient light (observation during bright moonlight periods possible)
\item Low power consumption ($\leq$50$\mu$W/mm$^2$)
\item Low operation voltage (typically $\sim$20 - 100 Volts)
\item No need for HV, as in PMTs
\item High Photon detection efficiency
\item Insensitivity to magnetic fields
\item Being a fairly new technology it is steadily improved, meaning a new generation of SiPMs every ~5months
\item Rapidly decreasing cost per mm$^2$
\end{enumerate}

The currently most challenging astrophysics experiment Silicon Photomultipliers are considered for is use in Cherenkov telescopes, with very demanding expectations in terms of photon detection efficiency. The photon detection efficiency quantifies the absolute efficiency of any photon detector to absorb a photon and produce a measurable signal at its output. To achieve a high photon detection efficiency in the 450nm regime, the design moves to very thin implantation layers on the surface in order to minimize the absoprion of shorter wavelength photons in insensitive areas. Figure(\ref{SiPM_scheme}) Different entry window coatings and avalanche structures (trenches etc.) explore the capable enhancements in the blue sensitive UV region.

\subsection{Gain of a Silicon Photomultiplier}
\label{subsec:SiPMGain}
The gain of a Silicon Photomultiplier measures the internal conversion of a photon incident into a signal at the output. The amplification of the device is expressed as the average number of charge carriers produced. There is no distinction weather the incident was caused by a single original photon or a thermal electron.
\begin{figure}[h]
\begin{equation}\label{eq:1}
\begin{split}
M = & \frac{Q}{e} \\
Q = &  C * (V_{bias}-V_{breakdown})
\end{split}
\end{equation}
\label{fig:Gain_conversion_formula}
\caption{The gain (M) of a Silicon Photomultiplier results form the deposited charge (Q) of the pulse generated from one cell when it detects one photon, devided by the charge per electron (e). The charge deposited per event is proportional to the cells capacitance (C) and the supplied over-voltage $(V_{bias}-V_{breakdown})$. This results in the gain (M) in units of total number of charge carriers, usually in the several $10^6$ range.}
\end{figure}\\
Estimation can be done via the voltage mean of the 1 p.e. amplitude: (e.g. see Figure (\ref{fig:S12642_PS})). For a better understanding, stating the gain in units of mV per photoelectron or $\frac{mV}{p.e.}$ is more suitable, as it gives a direct correlation between detected photoelectron and expected voltage amplitude. Given the very narrow pulse shapes, using the average pulse amplitude and extracting FWHM as a time measure of the total charge flowing during discharge and using the formula:
\begin{figure}[h]
\begin{equation}
\begin{split}
Q(p.e.) = & C * (V_{bias}-V_{breakdown})\\
U(p.e.) = & \frac{Q(p.e.)}{t(FWHM)} * 50ohm
\end{split}
\end{equation}
\caption{Resulting in the expected event charge flowing during capacitor discharge. Given the bias-voltage and C as capacitance of one cell and the resistance of the quenching resistor, a conversion factor and the average amplitude per photoelectron can be extracted.}
\end{figure}\\
The gain is obviously higher the larger the cell capacitance or the higher the bias-voltage (eq \ref{eq:1}) will be. But increasing the bias-voltage also increases dark counts and crosstalk. 
The gain is also dependent on the temperature, mainly through the quenching resistor but also from the silicon bulk itself, at a certain bias-voltage decreasing as temperature rises. The quenching resistor is affected by a lowering of the electrical conductivity with rising temperature, in accordance to the Wiedemann-Franz law, stating that the ratio of electrical and thermal conductivity remains constant. The silicon bulk at rising temperatures underlies increased crystal lattice movement. This impinges charge transport by increasing the probability that carriers might impact on the lattice before the carrier energy has become large enough for continued ionization. In order to counteract this, the electric field must be increased by increasing the supplied bias-voltage so ionization is more likely. Doing this has drawbacks as discussed before. For application as a photon detector, keeping the gain constant is an inevitable step, otherwise the shifting gain leads to problems. To do that, either the bias-voltage need to be asjusted to match ambient temperature, leading to problems with varying dark counts and crosstalk. Or the surface temperature must be regulated to be kept constant. Although more challenging hardware would be required, the latter option has obvious advantages, keeping dark counts and crosstalk and more important the gain constant by simply regulating the surface temperature.\\
Taking into account equation \ref{eq:1}, it appears that the breakdown-voltage can be estimated from the zero-crossing of a linear extrapolation of the gain at every bias-voltage per temperature. By doing this a linear breakdown-voltage dependence of the temperature can be observed. Appendix Figure(\ref{app:Device_Vbr})\\
I note that the gain, when parametrized over over-voltage, is essentially temperature independent, as I will show later in chapter {\ref{sec:results_ch}}: Figures (\ref{fig:S13360_Gain}) , (\ref{fig:LCT57_Gain}) , (\ref{fig:LVR6_Gain}) , (\ref{fig:SensL_Gain})\\


\subsection{Thermally induced dark counts}
Still temp dependent when plotted vs overvoltage
Inside the Silicon Photomultipliers depleted region a dark pulse originates from thermal excitation of an electron to the conduction band. Without an event photon present to trigger the avalanche, it is still indistinguishable from a photoelectron pulse. These thermally generated carriers are observed along with with the signal from a real photoelectron, presenting an irreducable source of noise. The number of dark pulses observed is refered to as dark counts ant the number per second as the dark count rate. For applications that need to operate in an environment with low noise, those dark counts are a concern. In IACT application of Silicon Photomultipliers however, this is only a minor problem, since IACTs operate in a naturally noisy environment. Even though sky darkness is one of the prime criteria of the proposed site selections for CTA, the surrounding Night Sky Background (NSB) at the most darkest side in Namibia will still exceed any random noise in the detector. The pollution of those NSB photons is unavoidable noise and will essentially limit the low energy resolution of the telescope to the NSB rate. As long as any random noise, being dark counts or other, is significantly below the NSB rate, it will not affect the telescopes performance.\\
Telescope performance simulations of the Schwarzschild-Couder MST of CTA at the site of Namibia showed a rate of $\sim$ 20 - 80 MHz per pixel for one of the older SiPM iterations from Hamamatsu. \cite{SiPMvsMAPMT} This is purely for NSB photons and pixels with a size of 6x6mm$^2$, the covered range originates from differences in illumination level of the night sky by galactic- and extragalactic-fields. 
I note that the DCR, when parametrized over over-voltage, remains temperature dependent, as I will show later in chapter {\ref{sec:results_ch}}: Figures (\ref{fig:S13360_DCR}) , (\ref{fig:LCT57_DCR}) , (\ref{fig:LVR6_DCROCT}) , (\ref{fig:SensL_DCROCT})\\


\subsection{Avalanche-induced secondary effects}

\begin{figure}[t]
\begin{centering}
%L, B, R, T
\resizebox{0.6\columnwidth}{!}{\includegraphics[trim=0cm 0cm 0cm 0, clip=true]{D:/OwnCloudData/00_WriteUp/04_Thesis/Pic/SiPM_Physics/{CT.JPG}}}
\caption{Secondary effects (bright red) caused by primary avalanches (dark red) in a Silicon Photomultiplier. In this paper a single pixel, in this figure, is referred to as a cell (see \ref{sec:SiPM}). Everything labeled under 1 is associated with prompt cross-talk, afterpulsing labeled as 2a, and delayed cross-talk labeled as 2b. Image adapted from \cite{ModelCTAP}}
\label{fig:correlated_noise}
\end{centering}
\end{figure}

An avalanche originating from the primary cell can sometimes, either directly or by reflection, propagate outside the cell and trigger an avalanche almost simultaneously in a secondary cell. This will, unless accounted for, degrade the Silicon Photomultipliers photon counting resolution, since the signal will be a merge of cross-talk cells and real incident cells. The probability for this is referred to as the cross-talk probability and the effect as the Optical Cross Talk, since it is conveyed via secondary photons generated in the primary avalanche. Afterpulsing also falls under this category, with the main difference to cross-talk being that the carrier triggers a secondary avalanche in the primary cell, basically generating a parasitic pulse inside the previously fired cell. Contained in a single cell, afterpulsing increases the measured charge registered for an incident photon. The difference in arival time of the secondary avalanche distinguishes different components comprising the cross-talk and afterpulsing. Those secondary avalanches can again emit photons, that can trigger secondary avalanches themselves, leading to high amplitudes, even in dark conditions. \\
Figure (\ref{fig:correlated_noise}) shows different physical processes causing secondary effects in Silicon Photomultipliers based on their delay time. Cause of the delay time is the dependence on the penetration depth of the incident photon, or the region the dark count generated, and the diffusion time inside the substrate. At long delay times afterpulsing and cross-talk are not distinguishable. The prompt Optical Cross Talk happens basically simultaneous to the primary avalanche, since it is unaffected by the primary cells recovery time and is labeled 1 in Figure (\ref{fig:correlated_noise}). It is either triggered by the secondary photon directly (1a) reaching the neighboring cell, or after first reflecting on the surface layer (1b) or the bottom surface (1c). If the cross-talk avalanche delay time is shorter than the detection resolution, the difference in signal between an Optical Cross Talk event or an incitend photon being detected, is not observable.\\
Time delayed Optical Cross Talk is caused by a carrier generated in the non-depleted substrate diffusing to a neighboring cell and triggering an avalanche in the depleted region (2b). If the carrier stays in the primary cell, the triggered avalanche is labeled afterpulsing (2a) and will have a lower amplitude, due to the cell not being recovered yet. The delay time is influenced by how deep the carriers are being trapped in the substrate, the time they need to diffuse to the surface, and also distinguished by traps with different lifetimes.\\
This is very important in IACT performance, since this effect gives random NSB and dark count photons the ability to rise to arbitrarily large amplitudes. The consequent need to raise the trigger threshold to counteract the resulting rising accidental-triggerate has a negative impact on the low energy resolution of the telescope.\\
Parametrisized with over-voltage, the secondary avalanche effects are temperature independent, shown in chapter {\ref{sec:results_ch}}: Figures (\ref{fig:S13360_OCT}) , (\ref{fig:LCT57_OCT}) , (\ref{fig:LVR6_DCROCT}) , (\ref{fig:SensL_DCROCT})\\

%\clearpage
\subsection{OCT dependency on cellsize}
\begin{figure}[h]
\begin{centering}
%L, B, R, T
\resizebox{0.7\columnwidth}{!}{\includegraphics[trim=0cm 0cm 0cm 0, clip=true]{D:/OwnCloudData/00_WriteUp/04_Thesis/Pic/SiPM_Physics/CellSize/OCT_vs_OV}}
\caption{Results of the Optical Cross Talk of 2 sets of 2 similar HPK S13360 devices, that only differ in their respective cell-size. HPK S13360 is the first device incorporating physical trenches in the upper layer, optically isolating each cell. Consequence to this is a drastic reduction in prompt cross-talk. Delayed cross-talk and afterpulsing are basically unafected. Upscaling the 25$\mu$m results shows an overlap between the 2, see text.}
\label{fig:cell-size}
\end{centering}
\end{figure}
Cross-talk is dependent on the ability of the secondary photons to reach a neighboring cell. This means, that an increase in cell-size and therefore cell-area should directly correlate with the Optical Cross Talk of a pixel. In Figure (\ref{fig:cell-size} the results of the complete Optical Cross Talk of 2 sets of 50$\mu$m and 25$\mu$ HPK S13360 devices are shown. Plotting the results of 2 similar devices, only different in their cell-size, and then multiplying the 25$\mu$m results by the factor derived from their difference in area, here 4, a correlation is visible. Scaling up the Optical Cross Talk of the 25$\mu$m cell, shows an overlap between the 2 cell-sized pixels. This means, that the Optical Cross Talk is directly area and therefore cell-size dependent. Research by J. Rosado and S. Hidalgo \cite{ModelCTAP} on the cross-talk probability of Hamamatsu SiPMs showed through Monte Carlo simulation, that \"the prompt crosstalk mostly takes place in a small area of pixels ($\sim$ 8) around the primary one.\" Which means, that the cross-talk is directly increasing with increasing cell-size, or in other words: with chance to diffuse to a neighboring cell. Small cells reduce the chance, as there is less area to a neighboring cell to pass through.\\
Since the measurements conducted by me do not differ between the range of secondary avalanche effects, the Optical Cross Talk shown contains every aspect, prompt and delayed as well as afterpulsing.


\subsection{Photon Detection Efficiency}
The Photon Detection Efficiency is the probability of a detector to absorb an incoming photon and produce a measurable signal at its output, and depends on a number of factors. First the photon must enter the depleted region via transmission through the surface of bare silicon, which has a reflectivity of 30$\%$. However, the transmission probability can be improved by coating the surface with a substrate with adequate thickness and a refraction index between air $\eta_{air} = 1.$ and silicon $\eta_{silicon} = 3.4$. Devices presented in this paper are coated with epoxy, a silicon resin and glass. The coating also has the added benefit of insulating and protecting the cells against environmental influence. A possible negative effect of coating is an increase in prompt cross-talk and a larger dependency of the overall cross-talk on the cellsize. The second factor is quantum efficiency, describing how susceptible the depleted region is to photons exciting electrons from the valence band to the conduction band. This is sometimes refered to as spectral response of a detector to reflect the wavelength dependence of a detector and makes the PDE wavelength dependent. The over-voltage dependency of the PDE is conveyed by the third factor, the avalanche probability. It depends on the electric field present, and thus on the applied over-voltage. The last factor is the fill-factor: the more the surface area of the detector is covered with active-area cells and the less dead-area exists between cells the higher the fill-factor is. \\
The Photon Detection Efficiency is commonly measured by illumination of the pixel and a calibrated reference photodiode with a flashing light source, and determining the average number of photons hitting the photosensor during a light pulse.

\begin{figure}[h]
\begin{equation}
PDE = \frac{N_{detected\:photons}}{N_{total\:photons}}
\end{equation}
\label{PDE_eq}
\caption{Photon Detection Efficiency is the percentage of detected photons versus overall emitted photons. }
\end{figure}



%______________________________________________________________________________________________________________________________________________________________________________________________________________
\clearpage
%\newpage
\section{\Large Experimental Setup}
\label{sec:exp_setup}
\begin{figure}[h]
\begin{centering}
%L, B, R, T
\resizebox{0.8\columnwidth}{!}{\includegraphics[trim=0cm 0cm 0cm 0, clip=true]{D:/OwnCloudData/00_WriteUP/04_Thesis/Pic/Setup/{setup_pic}.jpg}}
\caption{Will be updated. With Alphabetical captions for the different steps and references later in the text and setup scheme below, Picture of the inside of the thermal chamber is missing, will do later, and try to fit it next to this one}
\label{fig:Setup_Pic}
\end{centering}
\end{figure}
\begin{figure}[h]
\begin{centering}
%L, B, R, T
\resizebox{0.8\columnwidth}{!}{\includegraphics[trim=0cm 7.5cm 0cm 0cm, clip=true]{D:/OwnCloudData/00_WriteUP/04_Thesis/Pic/Setup/setup}}
\caption{Experimental setup scheme, Annotations see text}
\label{fig:Setup_Scheme}
\end{centering}
\end{figure}


The experimental setup in general is designed to house a variety of SiPM devices. Over the course of this paper, 5 different types of SiPMs were mounted on the setup and evaluated. It involves a thermal chamber Figure(\ref{fig:Setup_Scheme})(A) for temperature regulation which proved light tight and thus also serves as a dark box, to prevent any stray light to reach the SiPMs surface. Depending if the SiPM Figure(\ref{fig:Setup_Scheme})(A1) in testing is pre-manufactured on a test-array or supplied as a standalone chip, it is either mounted directly on a mechanical arm inside the chamber in the former case, or in the latter the mechanical arm supports a specifically designed PCB connecting to the device. Via the mount, bias-voltage is supplied and signal is transfered to the shaper Figure(\ref{fig:Setup_Scheme})(A2). !!!!!Need some more details on shaper\\\\
In some cases the output signals amplitude is to low to trigger the oscilloscope, therefore amplification is needed. I used an amplifier from MiniCircuits ZFL-1000-LN+ SN:283401542  Figure(\ref{fig:Setup_Scheme})(A3) supplied with different voltages depending on the tested device to amplify the shaped signal.\\\\
Data acquisition in the Laboratory is realized by a Lab-PC Figure(\ref{fig:Setup_Scheme})(B) , that forms the central control station for multiple pieces of equipment. It is connected to the oscilloscope Figure(\ref{fig:Setup_Scheme})(C), which records the waveforms of the device in testing, and then sends the data back to the PC via ethernet. The oscilloscope is a Lecroy XXXXX SN capable of 2.5 $\frac{GS}{s}$. The power supplies Figure(\ref{fig:Setup_Scheme})(D) control the bias-voltage of the Silicon Photomultipliers, ramping of the bias-voltage is controled by the PC. Lastly the thermal chamber is connected to the Lab-PC, which controls the temperature, and continuously rechecks it during temperature ramps. Signal transfer from the shaper to the oscilloscope is via a throughput on the side. All equipment is connected via ethernet, plugged into a common hub, to form a local network. While the temperature of the thermal chamber is ramping from the previous to the next set-point, the data is send to the Lab-PC.\\\\

Temperature regulation is an issue in the teststand, as there is no way of controling the SiPMs surface temperature. In dark conditions however, without conducting illumination tests, the shift in temperature on the SiPMs surface is only minimal. (Find the breakdown-voltage dependence on temperature in the appendix (\ref{app:VBR_table}). Checking the surface temperature of all devices with a temperature probe during testing showed minimal rising temperatures. So the influence of the temperature on the breakdown-voltage is only of minor concern. However, once illumination tests begin, the rising temperature on the SiPMs surface will no longer be negligible and the temperature must be regulated, either by cooling of the surface or including the temparature parameter.



%______________________________________________________________________________________________________________________________________________________________________________________________________________

\clearpage
%\newpage

\section{\Large Data Analysis}
\addtocontents{toc}{\protect\setcounter{tocdepth}{2}}
The analysis of the Silicon Photomultiplier waveforms is done exclusively in python following the sequence:\\
(Possibly to much detail in here)
\begin{enumerate}[topsep=0pt,itemsep=-1ex,partopsep=1ex,parsep=1ex]
\item data conversion
\item pedestal subtraction
\item peak detection
\item gain extraction
\item calculation
\end{enumerate}
The waveform data, after being transfered to a seperate PC, is analyzed offline. The goal is to extract event-data from the waveforms in order to produce pulse-area spectra for every bias-voltage at every temperature. To that end, the pedestal of the electronic noise must be found and subtracted from the data. After that, event-pulses are detected and integrated. This generates a list of pulse-areas, which is in turn used to fill a pulse-area histogram. To this pulse-area histogram, a model is fitted from which the gain can be extracted. Due to the linear behavior of the gain with rising bias-voltage, a regression line is fitted, from which the Dark Count Rate and the Optical Cross Talk are calculated using the original pulse-area histogram.
\subsection{Tracefile Conversion}
The oscilloscope produces the waveform data in its intrinsic data format, called a trace with the data suffix \.trc in binary. A trace file contains a header and two binary lists, an amplitude based on the oscilloscopes voltage-range and offset, and a list of the same length containing the associated event-time, based on the time-range and horizontal offset. The first step is therefore a conversion of the amplitude and associated event-time of all segments of a waveform trace file into two lists of floats. 
\clearpage
\subsection{Pedestal Subtraction}
\begin{wrapfigure}{R}{0.5\textwidth}
\centering
\includegraphics[width=0.5\textwidth]{D:/OwnCloudData/00_WriteUp/04_Thesis/Pic/Analysis/{HAM_T22.0_Vb68.5.trcFiltered1Zoom}.pdf}
\caption{\label{fig:PedSub}Raw, real data from a HPK SiPM in grey, in blue the pedestal subtracted and smoothed `'Filtered Signal 1''}
\end{wrapfigure}
A single waveform from the oscilloscope is anticipated to be uncentered fig(\ref{fig:PedSub})(grey), it will be slightly above or below zero, depending on the device setup (some devices produce inverted signals). The signal is mixed with electronic noise when it is observed and forms a pedestal, shifting the mean of the waveform from its original position to zero. Pedestal subtraction removes this average noise.\\
The first step of the process is reading in the uncentered waveform fig(\ref{fig:PedSub})(grey) and calculating an initial mean(mean0), expected to be slightly higher than the actual mean of the noise, due to the presence of event-pulses. The waveform is then shifted to about zero, by subtracting the mean0. A second mean of the now nearly centered waveform is taken (mean1). Now a new, same-sized array is formed and filled with the data from the waveform that is smaller than mean1, this represents the negative part of the noise. The data larger than mean1, the positive noise, is also filled into the array, but is negative-signed. This creates an array of the waveform centered around zero and, above the mean1, folded towards the negative. It proved reliably, that calculating the root-mean-square of this helper-array is a solid possibility of stripping the waveform of event-peaks. Taking the root-mean-square with a factor 3 of the helper-array, a cut is now applied to the waveformon on both, the positive and negative side. In that fashion, the remnant is the waveform between the positive and the negative root-mean-square and is now called a peakless-signal, representing the noise of the waveform.\\
Pedestal subtraction is done by calculating the mean of this peakless signal and subtracting it from the waveform. After that, the peakless signal is also smoothed by convolving it with a wide-windowed gaussian and subtracted from the waveform to eliminate any slow moving noise, the resulting waveform is called "Filtered Signal 1" fig(\ref{fig:PedSub})(blue).


\clearpage
\subsection{Peak Detection}
\begin{wrapfigure}{R}{0.5\textwidth}
\centering
\includegraphics[width=0.5\textwidth]{D:/OwnCloudData/00_WriteUp/04_Thesis/Pic/Analysis/{HAM_T22.0_Vb68.5.trcFiltered2Zoom}.pdf}
\caption{\label{fig:PeakDet}''Filtered Signal 1'' in grey before smoothing with a narrow gaussian to generate `'Filtered Signal 2'' in blue, which is used for peak finding.}
\end{wrapfigure}
Peak detection exploits the fact, that the first derivative of an event-peak will cross zero into the negative at the time of the peak maximum. The presence of random noise in the signal however will lead to many false detections. Therefore, before the detection of the event-peaks, the waveform fig(\ref{fig:PeakDet})(grey) is smoothed with a narrow-window gaussian with a width of about the FWHM of the devices characteristic event-pulse, in order to attenuate non-event peaks fig(\ref{fig:PeakDet})(blue). After the first derivative of the signal is calculated, which in python is a fast process if using arrays, a number of parameters decide the validity of the detected peaks. Most important parameters are a certain predetermined minimum amplitude, called the amplitude threshold or: minimum peak height. This is right now determined by eye, but could later be calculated based on the noise level. The second important parameter defining validity is the minimum peak distance, which defines how close two events can occur after another. The value is determined by the FWHM of the device in testing, which I expect to be sensible enough to resolve two events happening close after another. The peak detection algorithm can not distinguish between instantanious and delayed Oprical Cross Talk, but nonetheless, due to the fact, that the signal data is taken over many micro-seconds, all events are detected, independent of their source. On the other hand, this also means, that it is possible for two events to happen at the same time, for example a real photoelectron-event coinciding with delayed Optical Cross Talk. This can not be destinguished and will lead to a slight shift of the amplitude. (because delOCT is partial 1p.e.)   

\clearpage
\subsection{Gain Extraction}
\begin{wrapfigure}{R}{0.45\textwidth}
\centering
\includegraphics[width=0.45\textwidth]{D:/OwnCloudData/00_WriteUp/04_Thesis/Pic/GainFit/{PAFit_Annotated}.png}
\caption{\label{fig:PAFit}Pulse Area Histogram of a HPK S12642 with 1p.e. and $\Delta$p.e. positions. Multi gaussian as black dashed line.}
\end{wrapfigure}

The detected peaks are integrated with a window extending symmetricaly from the peak maximum, the width is chosen as slightly wider than the peaks FWHM. The generated list of peak areas is generating a peak area histogram, which is fitted with multiple gaussians using iMinuit in python , seen in Figure (\ref{fig:PAFit}). Two parameters are extracted from the fit to the peak area histogram. The first, the position of the 1p.e. peak is the position of the maximum of the first peak in the histogram, and the position of corresponding multiple p.e. events should be integral factors of the 1p.e. position. This proved to not be the case for some devices, the suspected source of this error is a pedestal generated during the peak integration. I studied the effect of the integration window in more detail in chapter \ref{subsec:subsec_mpd}.\\

\begin{wrapfigure}{R}{0.45\textwidth}
\centering
\includegraphics[width=0.45\textwidth]{D:/OwnCloudData/00_WriteUp/04_Thesis/Pic/GainFit/{GainFit_delta_pe}.png}
\caption{\label{fig:Gain_P}1p.e. position and $\Delta$p.e. extracted from the Pulse Area histogram at every bias-voltage for HPK S12642 with their respective regression lines.}
\end{wrapfigure} A second parameter is extracted from the peak area histogram to deal with this problem, which is the distance between N p.e. peak maxima of the histogram, labeled $\Delta$p.e. , which defines the gain. The apparent pedestal of the pulse area histogram makes extraction of the two parameters necessary in order to calculate the Dark Count Rate and the Optical Cross Talk. The gain of a Silicon Photomultiplier has a linear dependence of the supplied bias-voltage. Given that the bias-voltage range is deliberately chosen such that the over-voltage ranges from about 1V growing upwards, the range of linearity only starts at around the point of operation given by the manufacturer of the device. At the higher end of the bias-voltage range the behaviour usually starts to divert from the linearity. In order to get an estimation of the gain over a large range, both previoulsy extracted parameters, the 1p.e. peak and the distance $\Delta$p.e. are fitted with a linear regression line. The fit assumes linearity utilizing weighted least squares inherited from pythons statsmodels package. Fot the fit, only the data, where at least 3 gaussians are fitted to the peak area histogram is taken into account. Plotting both extracted parameters, as well as their respective regression lines versus bias-voltage, as in Figure (\ref{fig:Gain_P}), shows the difference between the two parameters. (SHOW FOR ALL DEVICES IN APPENDIX???). Comparing the manufacturer supplied brakdown-voltage from the datasheet for all devices showed, that the zero-crossing of the 1p.e. regression line is more consistent with datasheet values, in contrast to the zero-crossing of $\Delta$p.e. , which lies slightly higher. The over-voltage, corresponding to the set bias-voltage at any given temperature is calculated from this breakdown-voltage. 
Dark Count Rate and Optical Cross Talk are calculated utilizing 1p.e. position and $\Delta$p.e. derived from the regression line. Both values are applied, in the calculation, to the peak area histogramm with the Dark Count Rate being defined as:

\begin{figure}[h]
\begin{equation}
DCR = \frac{N_{events(>0.5p.e.)}}{T_{experiment}}
\end{equation}
\label{DCR_eq}
\caption{The Dark Count Rate of a Silicon Photomultiplier is defined as all events exceding 0.5 p.e. in apmlitude $N_{events(>0.5p.e.)}$ occuring over the experiment time $T_{experiment}$. Included in the measurement are thermally generated dark counts, as well as delayed cross-talk and afterpulsing with only a minor contribution.}
\end{figure}
,and the Optical Cross Talk as:
\begin{figure}[h]
\begin{equation}
OCT = \frac{N_{events(>1.5p.e.)}}{N_{events(>0.5p.e.)}}.
\end{equation}
\label{OCT_eq}
\caption{The Optical Cross Talk of a Silicon Photomultiplier is defined as all events exceding 1.5 p.e. in apmlitude$[N_{events(>1.5p.e.)]}$ devided by all events exceding 0.5 p.e. in apmlitude$[N_{events(>0.5p.e.)]}$. It scales with the number of photons produced inside an avalanche, as well as the probability of these photons to trigger a neighboring cell}
\end{figure}

\newpage
\subsection{Data Challenges}
\subsubsection{The influence of the minimum peak distance}
\label{subsec:subsec_mpd}
\begin{figure}[h]
\begin{centering}
%L, B, R, T
\resizebox{0.9\columnwidth}{!}{\includegraphics[trim=0cm 0cm 5cm 1cm, clip=true]{D:/OwnCloudData/00_WriteUP/04_Thesis/Pic/AnalysisSteps/1.8_A_S_C_REGR_IntWin+MPD/Regr/Compare_Regr_FixedGuess3mV520-55IntWinMPD100-63-25-13_DCR_vs_Vb.pdf}}
\caption{The difference between 4 minimum peak distance windows, in time-bins, during the peak detection. The dashed lines are the Dark Count Rates considering all recorded events, forming an upper limit. In pink, the effect of a different integration window is shown.}
\label{fig:MPD_plot}
\end{centering}
\end{figure}
The influence of the minimum peak distance is shown in Figure (\ref{fig:MPD_plot}) . Based on their bin-width, four different windows are tested. With the oscilloscopes sampling rate of 2.5$\frac{GS}{s}$, the windows of 100, 63, 25 and 13 time-bins, correspond to 40, 25, 10, 5 ns windows respectively. With a event pulse FWHM of ~10ns, setting the minimum peak distance to 100 bins, resulting in a 40ns window is visibly to large, as the algorithm will skip over valid data Figure (\ref{fig:MPD_plot})(red). After an event is detected, skipping over 40ns worth of data will result in errors of the Dark Count Rate, since the calculation uses the complete experiment time. Therefore a more reliable distance window must be chosen. The second window of 25ns was the next approach, originating from the length of the pulse-tail fig(\ref{fig:MPD_plot})(blue). This would lead to no detected events overlapping with the tail of one previouly detected, resulting in a sharper pulse-area spectrum. Compared to a window of approximately the pulse FWHM, the previously discussed window would still lead to lost event-data. Since the sharpness of the pulse-area spectrum is already sufficient, a window around the pulse FWHM was chosen as reference for all measured devices Figure (\ref{fig:MPD_plot})(green). Going lower that the pulse FWHM showed no improvement Figure (\ref{fig:MPD_plot})(black). 

\subsubsection{The influence of the integration window}
Figure (\ref{fig:MPD_plot}) also shows the influence of the size and shape of the integration window on the Dark Count Rate. The influence of the chosen integration window is most visible in their respective pulse area spectra. Choosing a narrow, symmetrical integration window of 5ns left and right of the peak maxima, the noise peak, or zero-peak, is much more prominent compared to the pulse area spectrum of a symmetrical 10 ns window. This leads to errors in the multi-gaus fitting step, or the fitting will fail alltogether. An asymmetrical integration window of 5ns left, 20ns right, to capture the influence of the pulse-tail proved, at first, to be the best solution as their was no visible zero-peak present. The low amplitude pulses are averaged out by the extended integration window to the right of the pulse-maximum. The downside of the asymmetrical window is the shifting of the puslse area spectrum and the fact, that the N p.e. peaks get blurred. The next step was widening the window on both sides. This proved the best solution, since there is no zero-peak visible and the N p.e. peaks are gaussian shaped. Please see the Appendix for the respective plots of pulse area spectra of the different windows \ref{subsec:PAS_window}.
\begin{figure}[t]
\begin{centering}
%L, B, R, T
\resizebox{0.9\columnwidth}{!}{\includegraphics[trim=0cm 0cm 5cm 1cm, clip=true]{D:/OwnCloudData/00_WriteUP/04_Thesis/Pic/AnalysisSteps/1.9_A_S_C_REGR_Thresh3-1p5mV/Regr/Compare_Regr_FixedGuess1p5mV-3mV520IntWinMPD100-63-25-13_DCR_vs_Vb}}
\caption{The difference between 4 minimum peak distance windows, in time-bins, during the peak detection. The dashed lines are the Dark Count Rates considering all recorded events, forming an upper limit. In pink, the effect of a lowered peak detection threshold is shown.}
\label{fig:PF_Thresh_plot}
\end{centering}
\end{figure}

\subsubsection{The influence of the peak detection threshold}

Choosing an adequate peak-finding threshold is a crucial step. Lowering of the threshold to $\sim$0.5 p.e. will cause the peak finding algorithm to misinterpret a lot of noise peaks as actual dark incidents. This, of course, leads to errors in the gain extraction and even if a gain regression line can be extracted, the resulting Dark Count Rates and Optical Cross Talk will be incorrect. Figure (\ref{fig:PF_Thresh_plot}), shows the effect a low peak finding threshold has on the Dark Count Rate.







%______________________________________________________________________________________________________________________________________________________________________________________________________________
\clearpage
\newpage
\section{\Large Results}
\addtocontents{toc}{\protect\setcounter{tocdepth}{2}}
\label{sec:results_ch}
\subsection{SiPM devices for CTA}

\begin{centering}
\begin{figure}[h]
\begin{tabular}{ |p{4.5cm} | c | c | c | c | p{1.5cm} | p{1.5cm} |}
    \hline
    Manufacturer                            & pixel size  & cell size & coating & connection & specifics               & pre-Amp       \\ \hline
    HPK S12642-1616PA-50  *CHEC-S SiPM               & 3$mm^2$     & 50$\mu$m  & SR      & TSV        &  no trenches & CHEC-S buffer \\ \hline
    HPK LCT5 S13360-6050CS                  & 6$mm^2$     & 50$\mu$m  & SR      & wire-bonds & trenches                & MS 13V        \\ \hline
    HPK LCT5 XXXXXXXXX                      & 6.915$mm^2$ & 75$\mu$m  & SR      & wire-bonds & trenches                & MS 8V         \\ \hline
    HPK 6050HWB-LVR-LCT                     & 6$mm^2$     & 50$\mu$m  & SR      & TSV        & trenches                & MS 13V        \\ \hline
    SensL FJ60035                           & 6$mm^2$     & 35$\mu$m  & glass   & TSV        & no trenches             & MS 15V        \\
    \hline
\end{tabular}
\caption{List of SiPM devices which results are presented in this paper. *(SR: silicon resin , MS: MiniCircuits , TSV: through silicon via)}
\end{figure}
\end{centering}
Silicon Photomultiplier devices for CTA are researched by many different groups, validating different characteristics. Besides the current SiPM for CHEC-S, newly developed prototypes offer a diverse range of pixel- and cellsizes. The majority of the devices are tested in Japan, at the University of Nagoya, conducting in depth analysis of the characteristics over a wide over-voltage range at one static temperature, mainly focusing on PDE and OCT, and their correlation. This correlation of OCT and PDE for all devices determines the candidate for CHEC-S, by comparing the highest PDE at the lowest possible OCT for each device. At MPIK, chosen candidate devices are examined regarding their temperature dependence and as to assist in the final decision by confirming results with a different analysis technique.\\
The chosen devices include the current CHEC-S device HPK S12642-1616PA-50, from Hamamatsu Photonics K.K., to be implemented in the first prototype camera. It is a previous generation SiPM, which was decided for use in 2014.(, due to the limited availability in high PDE devices at that time.) The manufacturer supplies a 16*16 channel premounted tile of 3mm$^2$ pixels with a cell-size of 50$\mu$m. To emulate the usage in a TARGET module, 4 3mm$^2$ pixels are electrically connected to form a 6mm$^2$ superpixel. The tile is typically coated with epoxy resin, but due to specific requirements regarding the uniformity of the coating, it was replaced with a very thin layer of silicon resin equivalent to later prototypes. The CHEC-S devices electronical connection is realised via the new through-silicon-via (TSV) concept of running a connecting solder through the silicon bulk, instead of wiring on the outside, greatly increasing the fill-factor, but also including some disadvatageous sideffects later shown.\\
Additional tested devices include the LCT5 generation from Hamamatsu Photonics K.K., so called due to their low cross talk properties, namely the commercial available S13360-5050CS. This device is the first to include physical trenches between the cells, effectively dividing the cells optically and thus reducing prompt cross-talk probability.\\
The LCT5 generation also made two prototype devices avaiable for testing at MPIK, the first, HPK LCT5 SN CATANIA, is a larger iteration with a cellsize of 75$\mu$m and a pixelsize of 6.915mm$^2$. Being from the LCT5 generation it also includes optical trenches. The second device available is a LCT5 prototype designated 6050HWB-LVR-LCT , LVR meaning low-voltage-range, with the same physical dimensions and properties as the S13360 device but incorporating TSV technology and possibly other unknown deviations.\\
The final device is a commercially avaiable test-array designated FJ60035 from SensL, premounted on a test-array by the manufacturer. It has the same pixel-size, but a much smaller cell-size 35$\mu$m than the previous mentioned devices and a different coating (glass).\\

The tests conducted in Nagoya contain many different iterations of the LCT5 generation with variing pixel- and cell-sizes.\\
For a full overview of the considered Silicon Photomultipliers, please refer to the appendix under (\ref{appendix:SiPM_Nagoya}).



\clearpage
\subsection{Hamamatsu S12642-1616PA-50 50$\mu$m 3mm}
\begin{wrapfigure}{R}{0.3\textwidth}
\centering
\includegraphics[width=0.3\textwidth]{D:/OwnCloudData/00_WriteUp/04_Thesis/Pic/SiPM_Pics/CHECSPIC.JPG}
\caption{\label{fig:CHECSTILE}CHEC-S tile}
\end{wrapfigure}
The Silicon Photomultiplier by Hamamatsu Photonics designated S12642-1616PA-50 is a 3 mm by 3 mm device. The array uses the through-silicon-via technology, meaning there are no wire-bonds present, the electrical connection is realised through the silicon-body. The pixels are coated with a thin film of silicon resin, after the previously used epoxy resin proved not uniform enough. One array consists of 256 pixels, 4 of which are electrically tied togother to form a 6mm by 6mm superpixel respectively. This practice is necessary for the pre-production camera CHEC-S, because the focal plane is mechanically designed to house 64 6mm$^{2}$ pixels, connected to the TARGET modules. Furthermore I expect this to have an influence on my results due to electrical crosstalk, but this is only of minor concern due to the following. 


\begin{figure}[h]
\begin{centering}
%L, B, R, T
\resizebox{0.8\columnwidth}{!}{\includegraphics[trim=0cm 0cm 0cm 0, clip=true]{D:/OwnCloudData/02_Results/HPK_S12642/{Average_25.0_67.8_PulseShape}.pdf}}
\caption{The average pulse shape of the 1photoelectron in blue and the 2photoelectron pulse in red of HPK S12642 at 25$^{\circ}$~C and 67.8V, which is around the proposed operating point. Both pulses are averaged over >>1000 events and normalized to illustrate possible differences in pulseshape resulting from the utilized shaping electronics. Both pulses have a FWHM of around 10ns and are nearly free of ringing. The resulting average amplitude of the 1p.e. pulse is later used to calculate the Gain in [mV/p.e.] instead of [V*IntWin] by cross-referencing the 1p.e. amplitude at multiple bias-voltages.}
\label{fig:S12642_PS}
\end{centering}
\end{figure}
My measurements of the CHEC-S tile concentrate on the array as an as-is device. This means all results, influenced by external factors outside the actual Silicon Photomultipliers physics, are valid on the assumption, that the way I was conducting the measurements is the way the Photomultiplier will later be incorporated into the camera. On that ground, deviations of my results from the results of other groups and the manufacturer itself are expected. To clarify this further, I expect, for example, that the tests done at Hamamatsu Photonics where conducted on a single 3mm$^2$ pixel, not an array of 256 pixels, where 4 are tied together. Also divergence of shaping and amplification electronics between the groups will result in some differences. For this test, the CHEC-S tile is connected to the CHEC-S buffer, supplied with $\pm$4V, where the signal is amplified. This signal in turn is then shaped via the CHEC-S shaper, developed by the University of Leicester. This is done in order to lower the unshaped pulse from a FWHM in the ~100s ns to ~10ns. This whole amplification and shaping chain is simulating later usage in the TARGET modules. I conducted the measurements multiple times on different pixels of the CHEC-S tile (appendix{\ref{app:CHEC_S_multipixel_Gain}}{\ref{app:CHEC_S_multipixel_DCR}}{\ref{app:CHEC_S_multipixel_OCT}}).


%\newpage
\subsubsection{Gain}
As descibed in section{\ref{subsec:SiPMGain}}, the average pulse shape fig(\ref{fig:S12642_PS}) is used to convert the relative gain from the analysis procedure to an absolute gain in sensible units. This is necessary, because the analysis aims to use pulse-area rather than -height, which results in this relative gain being in units of $V*timebins$, instead of plain V. In Figure (\ref{fig:S1264_Gain}) (left) the relative gain is shown, the right side shows the gain after conversion.
A lower gain with increasing temperature is expected and described in detail in the chapter {\ref{subsec:SiPMGain}}. In short, increased lattice movement due to higher temperature hinders photoelectron transport. The effects visible at extreme bias-voltages at both ends are partially analysis related. The gain of a SiPM is expected to be linear over bias-voltage at a constant temperature. 

\begin{figure}[h]
\begin{centering}
\begin{overpic}[width=0.4\columnwidth,trim=0cm 0cm 0cm 0, clip=true,tics=10]{D:/OwnCloudData/02_Results/HPK_S12642/Paper/S12642Gain_vs_Vb_Combined}
%\put(20,85) {X Datasheet value}
\end{overpic}
\resizebox{0.4\columnwidth}{!}{\includegraphics[trim=0cm 0cm 0cm 0, clip=true]{D:/OwnCloudData/02_Results/HPK_S12642/Paper/S12642Gain_vs_OV_Combined}}
\caption{Gain of the HPK S12642 pixel, plotted against over- , bias-voltage and temperature. }
\label{fig:S12642_Gain}
\end{centering}
\end{figure}

%\begin{figure}[h]
%\begin{centering}
%\resizebox{0.4\columnwidth}{!}{\includegraphics[trim=0cm 0cm 0cm 0, clip=true]{D:/%OwnCloudData/02_Results/HPK_S12642/Paper/S12642Gain_vs_Vb_Combined}}
%\resizebox{0.4\columnwidth}{!}{\includegraphics[trim=0cm 0cm 0cm 0, clip=true]{D:/OwnCloudData/02_Results/HPK_S12642/Paper/S12642Gain_vs_OV_Combined}}
%\caption{Gain of the HPK S12642 pixel, plotted against over- , bias-voltage and temperature. }
%\label{fig:S12642_Gain}
%\end{centering}
%\end{figure}

In the lower regime at Vb ~66.5V my analysis method struggles to pick up pulses, because of the very low gain compared to the noise. Depending on the chosen peak-finding threshold, I expect the analysis to interpret noise peaks as 1p.e. peaks at an increasing rate, the lower the overvoltage is. This is visible in the sudden break in linearity at 30$^{\circ}$~C and 35$^{\circ}$~C , where the gain is almost in a plateau, due to this effect. The roll-over of the gain at the highest bias-voltages is in part a result of a voltage drop across the bias resistor occuring, because of high current flow through the SiPM. At higher temperatures, and therefore higher dark rates, the effect occurs at lower over-voltages. A second influence is caused again, by the noise at Vov ~5V, which is very high compared to the proposed point of operation at Vov ~3V. The same threshold is again counting the now increased noise peaks as 1p.e. peaks, but due to the abundance of 1p.e. pulses this just results in an apparent lowering of the gain.




%\newpage
\subsubsection{Dark Count Rate}
The Dark Count Rate is expected to increase with temperature, which is the case for S12642 showb in Figure (\ref{fig:S12642_DCR}). I also expect it to follow a nearly linear progression, a sudden turn-up or turn-down of the Dark Count Rate would be analysis related. The turn-up at a certain point is visible in Figure (\ref{fig:S12642_DCR}), particulary for 15$^{\circ}$~C (purple) and 20$^{\circ}$~C (blue) respectively. At 15$^{\circ}$~C and an Overvoltage of ~4V, the Dark Count Rate starts to deviate from the previously linear behaviour. It starts to rise more rapid than before, which I can attribute to the fact, that the Optical Cross Talk at that point is very high; higher than 50$\%$  fig(\ref{fig:S12642_OCT}) (left). I suspect that I do not reach this critical point for the higher temperatures of 25$^{\circ}$~C (green) and 30$^{\circ}$~C (red), so the effect is barely, if not at all visible. 
\begin{figure}[h]
\begin{centering}
%L, B, R, T
\resizebox{0.45\columnwidth}{!}{\includegraphics[trim=0cm 0cm 0cm 0, clip=true]{D:/OwnCloudData/02_Results/HPK_S12642/Paper/S12642DCR_vs_Vb_Combined}}
%\resizebox{0.45\columnwidth}{!}{\includegraphics[trim=0cm 0cm 0cm 0, clip=true]{D:/OwnCloudData/02_Results/HPK_S12642/Paper/S12642DCR_vs_OV_Combined}}
\begin{overpic}[width=0.45\columnwidth,trim=0cm 0cm 0cm 0, clip=true,tics=10]{D:/OwnCloudData/02_Results/HPK_S12642/Paper/S12642DCR_vs_OV_Combined}
\put(5,35) {Datasheet value}\put(42,31) {X}
\end{overpic}


\caption{Dark Count Rate of the HPK S12642 pixel, plotted against over- , bias-voltage and temperature. Datasheet value at operation voltage = 2.4V and 25$^\circ$C measured by current}
\label{fig:S12642_DCR}
\end{centering}
\end{figure}
At 35$^{\circ}$~C (yellow) I suspect my analysis is not able to count every pulse, due to the high rate of 9-10 MHz. A majority of the pulses overlapping each other and being counted as a 2p.e. event rather than 2 1p.e. events would reduce the Dark Count Rate significantly. At this point, heating of the Silicon Photomultipliers surface due to the high rate could also affect the Dark Count Rate through shifting of the temperature slightly upwards, away from 35${^\circ}$~C . So that the Dark Count Rate declared at 35${^\circ}$~C is in reality the rate at higher temperatures. \\\\
At the lower end of the bias-voltage range, I suspect, that a major part of the found 1p.e. pulses are actually noise related. So the Dark Count Rate changing to a plateau is expected. This is also due to the fact, that my measurements are done with a fixed bias-voltage range. Due to the increase of the breakdown-voltage with rising temperature, part of the measured bias-voltage range corresponding to a very low over-voltage, attributes to this effect. In order to reliably measure beyond an overvoltage of ~2.5V in the lower range, the noise would need to be improved.  





%\newpage
\subsubsection{Optical Cross Talk}
\begin{figure}[h]
\begin{centering}
%L, B, R, T
\resizebox{0.4\columnwidth}{!}{\includegraphics[trim=0cm 0cm 0cm 0, clip=true]{D:/OwnCloudData/02_Results/HPK_S12642/Paper/S12642OCT_vs_Vb_Combined}}
\resizebox{0.4\columnwidth}{!}{\includegraphics[trim=0cm 0cm 0cm 0, clip=true]{D:/OwnCloudData/02_Results/HPK_S12642/Paper/S12642OCT_vs_OV_Combined}}
\caption{Dark Count Rate of the HPK S12642 pixel, plotted against over- , bias-voltage and temperature.}
\label{fig:S12642_OCT}
\end{centering}
\end{figure}
The Optical Cross Talk should be linear and independent from temperature. This is confirmed for HPK S12642. Minor diviations from that are probably due to slight errors in the breakdown-voltage calculation from the gain regression line. The diviation of 30${^\circ}$~C and 35${^\circ}$~C below an over-voltage of 2V stems from the way the gain regression line is used to calculate both Dark Count Rate and Optical Cross Talk. At higher temperatures the lower voltage range is dominated by noise, so using the gain regression line to calculate the Optical Cross Talk at those low voltages leads to the visible effect of the first few datapoints of 30${^\circ}$~C and 35${^\circ}$~C. 
The deviations between the different groups results at 25${^\circ}$~C (green) are caused by 4 major contributions. Firstly the difference in the tested device. While I take measurements on 4 3mm$^2$ pixels electricaly tied together, the way the device will later be implemented into CHEC-S, the groups in the US and Hamamatsu Photonics are likely to run tests on 1 3mm$^2$ pixel only. Secondly I suspect a slight difference in amplification and shaping electronics. The measurements I conducted as well as the measurements of Leicester are done with the same shaper and buffer configuration. The difference here is, thirdly, my measurements are done with dark counts only, while measurements from other groups are conducted with a pulsed light source and reading out timed windows. This causes the results from Leicester to be difficult to compare against, their surface temperature of the SiPM is likely much higher than 25${^\circ}$~C, and thus, a misinterpreted breakdown-voltage at 25${^\circ}$~C causes a shift of the Optical Cross Talk to the right. Lastly the difference in actual data taking and analysis procedure must be mentioned, also this is only of minor concern, as we will see with other measured devices.






\clearpage
%\newpage
\subsection{Hamamatsu LCT5 50$\mu$m 6mm}
\label{subsec:LCT56}
\begin{wrapfigure}{R}{0.4\textwidth}
\centering
\includegraphics[width=0.4\textwidth,trim=0cm 2cm 0cm 0, clip=true]{D:/OwnCloudData/00_WriteUp/04_Thesis/Pic/SiPM_Pics/LCT56mmPIC.JPG}
\caption{\label{fig:S13360_pixel}HPK S13360 6050CS pixel}
\end{wrapfigure}

The Silicon Photomultiplier designated HPK S13360 6050CS fig(\ref{fig:S13360_pixel}) is an LCT5, meaning Low Cross Talk 5th iteration device from Hamamatsu Photonics. It is one of the most promising candidates for later usage in CHEC-S. It has a pixelsize of 6mm$^2$ consisting of 14400 cells with a cellsize of 50$\mu$m. The present device and its similar iterations are the first to incorporate trenches bordering each cell, effectively insulating the cells and reducing the prompt Optical Cross Talk very effectively. Tests are done with a single pixel only, in contrast to measurements done on S12642. It is mounted on a ceramic chip and coated with a silicon resin that is UV-transparent. Wire bonds supply the electrical contact. A similar, but not tested, device from the same generation uses through-silicon-via (TSV) technology, realising electrical connection through the silicon bulk, allowing a tighter fit of the cells, with minimal dead-space.



\begin{figure}[h]
\begin{centering}
%L, B, R, T
\resizebox{0.8\columnwidth}{!}{\includegraphics[trim=0cm 0cm 0cm 0, clip=true]{D:/OwnCloudData/02_Results/LCT5_50um_6mm/{Average_25.0_55.0_PulseShape}.pdf}}
\caption{The average pulse shape of the 1photoelectron in blue and the 2photoelectron pulse in red of HPK S13360 6050CS at 25$^{\circ}$~C and at point of operation. Both pulses have a  FWHM of around 5ns and ring for approximately 20ns with an undershoot of 20\%. }
\label{fig:S13360_PS}
\end{centering}
\end{figure}

 The layout of the single pixle test device made external amplification necessary. I used a MiniCircuits PreAMP, which was supplied with 13V during this test. Shaping of the pulse is conducted by a CHEC-S shaper, modified to fit the new unshaped pulse. The pulse shape fig(\ref{fig:S13360_PS}) makes the pulses appear much harder to analyze, due to the possibility of events occuring during the ringing window. This assumption proved untrue, due to the devices low Dark Count Rate and Optical Cross Talk, so the multi incident probability is also low.


%\clearpage
\subsubsection{Gain}


The gain of the LCT5 50$\mu$m 6mm device is clearly linear with some minor outliers at 30$^{\circ}$C.  The same effect as with S12642 is visible at 35$^{\circ}$C, again counting noise peaks as 1p.e. peaks, resulting in an apparent lowering of the gain and the slope changing over into a plateau. In Figure(\ref{fig:S13360_Gain})(left) the gain is shown, plotted against over-voltage. It is still dependant on temperature, but due to reliable breakdown-voltage calculation, the spread is much smaller than, if plotted against bias-voltage. The same conversion is done to transform relative gain into an absolute gain with sensible units. When parametrized with over-voltage, the gain is essentially temperature independent.
%\\\\\\\\\\\\\\\\\\\\\\\\\\ %wrapfigure spacing
\begin{figure}[h]
\begin{centering}
%L, B, R, T
% \resizebox{0.4\columnwidth}{!}{\includegraphics[trim=0cm 0cm 0cm 0, clip=true]{D:/OwnCloudData/02_Results/HPK_S12642/Paper/S12642RelGain_vs_Vb_Combined}}
\resizebox{0.4\columnwidth}{!}{\includegraphics[trim=0cm 0cm 0cm 0, clip=true]{D:/OwnCloudData/02_Results/LCT5_50um_6mm/Paper/LCT5_6mmGain_vs_Vb_Combined}}
\resizebox{0.4\columnwidth}{!}{\includegraphics[trim=0cm 0cm 0cm 0, clip=true]{D:/OwnCloudData/02_Results/LCT5_50um_6mm/Paper/LCT5_6mmGain_vs_OV_Combined}}
%\resizebox{0.4\columnwidth}{!}{\includegraphics[trim=0cm 0cm 0cm 0, clip=true]{./Fig/{Analysis_Page/filter2}.jpg}}
%\resizebox{0.4\columnwidth}{!}{\includegraphics[trim=0cm 0cm 0cm 0, clip=true]{./Fig/{Analysis_Page/peakpos}.jpg}} 
\caption{Gain of the HPK S13360 pixel, plotted against over- , bias-voltage and temperature. }
\label{fig:S13360_Gain}
\end{centering}
\end{figure}


%\newpage
\subsubsection{Dark Count Rate}
The Dark Count Rate of two similar HPK S13360 devices is shown in Figure(\ref{fig:S13360_DCR}). The bars show the difference between the two devices, the results of one device is used as a reference, while the deviation is illustrated with the filled bar. The Dark Count Rate of HPK S13360 followes the expected behaviour, mostly linear in the significant range and rising with increasing temperature. Below an over-voltage of 2.5V my analysis struggles, I suspect the gain, compared to noise, to be to low for my analysis to pick up pulses. Thus the regression line calculation is unreliable in this range. The turnup at high over-voltages is most prominent at 0$^{\circ}$C(teal) after an over-voltage of 9V. This is also the point where the Optical Cross Talk rises very rapidly.
\begin{figure}[ht]
\begin{centering}
%L, B, R, T
\resizebox{0.45\columnwidth}{!}{\includegraphics[trim=0cm 0cm 0cm 0, clip=true]{D:/OwnCloudData/02_Results/LCT5_50um_6mm/Paper/LCT5_6mmDCR_vs_Vb_Combined}}
%\resizebox{0.45\columnwidth}{!}{\includegraphics[trim=0cm 0cm 0cm 0, clip=true]{D:/OwnCloudData/02_Results/LCT5_50um_6mm/Paper/LCT5_6mmDCR_vs_OV_Combined}}
\begin{overpic}[width=0.45\columnwidth,trim=0cm 0cm 0cm 0, clip=true,tics=10]{D:/OwnCloudData/02_Results/LCT5_50um_6mm/Paper/LCT5_6mmDCR_vs_OV_Combined}
\put(0,35) {Datasheet value between}\put(33,19) {X}\put(33,39) {X}\put(34,29) {\bigg|}
\end{overpic}
\caption{Dark Count Rate of the HPK S13360 pixel, plotted against over- , bias-voltage and temperature. The Datasheet value at 25$^\circ$C of the DCR at $V_{OV}=3V$ covers a very large range to exactly compare against, and is measured by current}
\label{fig:S13360_DCR}
\end{centering}
\end{figure}

%\clearpage
\subsubsection{Optical Cross Talk}

Compared to measurements on HPK S13360 done at the Nagoya University, Japan fig(\ref{fig:S13360_OCT}) (faded green bar), I see a very strong correlation of the Optical Cross Talk in the over-voltage range between 2.5V and 9V. The Optical Cross Talk in this range is linear and independent from temperature, with minor deviations attributed to the breakdown-voltage calculation from the regression line, causing the horizontal shift. In contrast to my technique, using only dark counts, the measurements at Nagoya University followed a pulsed light source approach, reading out a time-window after the laser incident. This could have consequences, since Optical Cross talk with a large delay time could be missed. Deviations below an over-voltage of 2.5V are expected, they are very likely caused by the regression line calculation being unreliable in this range due to the analysis method struggling to pick up pulses using dark counts. Above an over-voltage of 9V, which is also the point of the turnup of the Dark Count Rate, the Optical Cross Talk is no longer linear and the deviation from the results of Nagoya University increase very rapidly. I suspect the rapid increase in both Dark Count Rate and Optical Cross Talk to be caused by the over-voltage reaching ranges, where interpretation of noise as a 1p.e. pulse becomes more likely. This, joint together with the usage of the MiniCircuits amplifier supplied with 13V, makes false interpretation of noise as pulses even more likely. I suspect these two reasons in conjunction are responsible for both, the sudden rise of the Dark Count Rate as well as the deviation of the Optical Cross Talk from linearity and the results of Nagoya University, above over-voltages around 9V. In summary, the correlation between the two measurements, conducted by two different methods of data acquisition and analysis, is evident.\\\\

The S13360 series is the first to incorporate physical barriers, called trenches, effectively insulating the cells from each other. This drastically reduces the promt cross-talk, while increasing the percentage the delayed cross-talk contributes to the overall cross-talk shown. This could also be the reason for the upturn compared to data from the University of Nagoya, at higher over-voltages the contribution from delayed cross-talk is higher. With the trenches effectively reducing the prompt cross-talk and the difference in analysis, the effect could be partially explanied by increased contribution of the delayed cross-talk. More on this subject in chapter {\ref{sec:comparison}}.


\begin{figure}[h]
\begin{centering}
%L, B, R, T
\resizebox{0.45\columnwidth}{!}{\includegraphics[trim=0cm 0cm 0cm 0, clip=true]{D:/OwnCloudData/02_Results/LCT5_50um_6mm/Paper/LCT5_6mmOCT_vs_Vb_Combined}}
\resizebox{0.45\columnwidth}{!}{\includegraphics[trim=0cm 0cm 0cm 0, clip=true]{D:/OwnCloudData/02_Results/LCT5_50um_6mm/Paper/LCT5_6mmOCT_vs_OV_Combined}}
\caption{Optical Cross Talk of the HPK S13360 pixel, plotted against over- , bias-voltage and temperature. }
\label{fig:S13360_OCT}
\end{centering}
\end{figure}





\clearpage
\subsection{Hamamatsu LCT5 75$\mu$m 7mm}
\begin{wrapfigure}{R}{0.4\textwidth}
\centering
\includegraphics[width=0.4\textwidth,trim=0cm 0cm 0cm 0.1cm, clip=true]{D:/OwnCloudData/00_WriteUp/04_Thesis/Pic/SiPM_Pics/LCT57mmPIC.JPG}
\caption{\label{fig:LCT57_pixel}HPK LCT5 7mm pixel}
\end{wrapfigure}

XXXXXXXXXXXX is a larger LCT5 prototype Silicon Photomultiplier of the same design as S13360-6050CS fig(\ref{fig:LCT57_pixel}). With an increase in cellsize to 75$\mu$m, the device gains a higher fill-factor than 50$\mu$m devices. The pixel-area is also expanded to 6.915mm$^2$, which will result in a higher fill-factor (less deadspace), so presumably a higher PDE but also a higher OCT. Since it is a prototype device, there is limited data from datasheets. The ID number suggests, that it is also a wire-bond device with a UV-transparent silicon-resin coating. It is also a single pixel test device, so external amplification is necessary with the MiniCircuits PreAMP, supplied with 8V during this test. The signal is also shaped by a differently modified CHEC-S shaper, which results in a pulse shape similar to S12642, but with a much lower amplitude fig(\ref{fig:LCT57_PS}).

%\newpage
\begin{figure}[h]
\begin{centering}
%L, B, R, T
\resizebox{0.8\columnwidth}{!}{\includegraphics[trim=0cm 0cm 0cm 0, clip=true]{D:/OwnCloudData/02_Results/LCT5_75um_7mm/{Average_25.0_55.0_PulseShape}.pdf}}
\caption{The average pulse shape of the 1photoelectron in blue and the 2photoelectron pulse in red of HPK LCT5 7mm at 25$^{\circ}$~C and at point of operation. Both pulses have a FWHM of around 7ns and an undershoot of 20\%, with no ringing. }
\label{fig:LCT57_PS}
\end{centering}
\end{figure}


\clearpage
\subsubsection{Gain}
\label{subsubsec:LCT57Gain}
\begin{wrapfigure}{R}{0.45\textwidth}
\centering
\includegraphics[width=0.45\textwidth,trim=0cm 0cm 0cm 0, clip=true]{D:/OwnCloudData/02_Results/LCT5_75um_7mm/Paper/LCT5_7mmGain_vs_OV_Combined}
\caption{\label{fig:LCT57_Gain}Gain of the HPK LCT5 7mm pixel}
\end{wrapfigure}
Figure(\ref{fig:LCT57_Gain}) shows the gain of the LCT5 7mm device. Two sets of measurements are done for 25$^\circ$C to extend the measured range. The first set of measurements covers the lower over-voltage range, where the low gain makes external amplification necessary with a MiniCircuits PreAMP supplied with 8V. I chose the lowest possible amplification of the PreAMP, so that reaching the point of saturation of the oscilloscope input is as late as possible as the over-voltage rises. Saturation of the oscilloscope occurs due to the possibility of generating very large p.e. (>10p.e.) events at the higher over-voltages, which are saturating the input. Joint together, the LCT5 7mm device and the MiniCircuits PreAMP at 8V reach this point at an over-voltage of $\sim$6V. In Figure(\ref{fig:LCT57_Gain}) the results from the lower range measurement are displayed as the lower-range green line extending between an over-voltage of 1.6V to 5.4V.
The configuration for the second test removes the PreAMP from the setup, which makes the lower over-voltage range unaccesible, but extends the range to higher over-voltages. This configuration reaches the point of saturation at an over-voltage of $\sim$8V. The higher range measurement results are displayed as the second green line (25$^\circ$C) extending from 3.4V to 7.2V over-voltage in Figure(\ref{fig:LCT57_Gain}). There is a clearly visible overlap of the two measurements between $\sim$3.4V and $\sim$5.4V . It also seems, that the gain dependency on temperature is reversed. While for all other devices, the gain lowers with rising temperatures, the LCT5 7mm device seems to show inversed behaviour. This inverse behaviour is caused by the calculation of the breakdown-voltage from the gain-regression line, with this and the bias-voltage, the over-voltage is calculated, causing a horizontal shift. Plotting the gain versus bias-voltage, however, shows the expected behaviour.


\newpage
\subsubsection{Dark Count Rate}
\label{subsubsec:LCT57DCR}
\begin{wrapfigure}{R}{0.4\textwidth}
\centering
\includegraphics[width=0.4\textwidth,trim=0cm 0cm 0cm 0, clip=true]{D:/OwnCloudData/02_Results/LCT5_75um_7mm/Paper/LCT5_7mmDCR_vs_OV_Combined}
\caption{\label{fig:LCT57_DCR}Dark Count Rate of the HPK LCT5 7mm pixel}
\end{wrapfigure}
The behaviour of the Dark Count Rate of the HPK LCT5 7mm device is shown in Figure(\ref{fig:LCT57_DCR}) and is as expected, in contrast to the behaviour of the gain of LCT5 7mm, as discussed in section {\ref{subsubsec:LCT57Gain}}. It follows a linear progression in the relevant range and increases with rising temperature. I suspect the over-voltage range above $\sim$2.5V to be relevant. The extended range measurement at 25$^\circ$C confirms this behaviour. LCT5 7mm shows a linear Dark Count Rate over an over-voltage range of 4V. The faded green bar in Figure(\ref{fig:LCT57_DCR}) shows results from measurements undertaken by the Department of Physics and Astronomy at the University of Catania. Those measurements were conducted on the exact same device, which is an important point, but with a different method of data acquisition and data analysis. Analysis techniques are discussed in chaper {\ref{sec:comparison}}. The correlation between the two experiments is evident, although there is differences in the acyquisition and analysis process. This is further proof for the relevancy of the analysis technique employed in this paper. 


\subsubsection{Optical Cross Talk}
\begin{wrapfigure}{R}{0.4\textwidth}
\centering
\includegraphics[width=0.4\textwidth,trim=0cm 0cm 0cm 0, clip=true]{D:/OwnCloudData/02_Results/LCT5_75um_7mm/Paper/LCT5_7mmOCT_vs_OV_Combined}
\caption{\label{fig:LCT57_OCT}Dark Count Rate of the HPK LCT5 7mm pixel}
\end{wrapfigure}

The Optical Cross Talk is expected to be linear and independent from temperature. This is the case in the, in section {\ref{subsubsec:LCT57DCR}} established, relevant over-voltage range of above $\sim$2.5V. Minor deviations are attributed to the calculation of the breakdown-voltage from the gain-regression line. The over-voltage is calculated from the former and the supplied bias-voltage, which in turn causes a slight horizontal shift. With that, comparing my results to the measurements from the University of Catania shows a strong correlation. Keeping in mind, that the process of data acquisition and analysis is different for both measurements further proofs the analysis technique valid. 



\clearpage

\subsection{Hamamatsu LVR 50$\mu$m 6mm}
\begin{wrapfigure}{R}{0.4\textwidth}
\centering
\includegraphics[width=0.4\textwidth,trim=40cm 25cm 35cm 25cm, clip=true]{D:/OwnCloudData/00_WriteUp/04_Thesis/Pic/SiPM_Pics/LVR6mmPIC.JPG}
\caption{\label{fig:LVR6_pixel}HPK LVR 6mm pixel}
\end{wrapfigure}

The Silicon Photomultiplier by Hamamatsu Photonics with the designation 6050HWB-LVR-LCT is a special prototype of the LCT5 design. LVR is an abreviation of Low Voltage Range, meaning the device is meant to be operated at much lower operation voltages than other LCT5 devices. It has the same physical size as an LCT5 50$\mu$m 6mm device (S13360 chapter {\ref{subsubsec:LCT56}}), a pixelsize of 6mm pixel with a cellsize of 50$\mu$m. The recommended point of operation however is $\sim$15V below that of the S13360 device, specifically at 40.2V(LVR) instead of 54.7V (S13360). The unshaped signal is similar to other LCT5 devices, therefore using the same modified CHEC-S shaper is feasible in this case, resulting in a similar pulse shape fig(\ref{fig:LVR6_PS}). After that the signal is amplified with the same MiniCircuits PreAMP supplied with 8.5V.
\\

%\newpage
\begin{figure}[h]
\begin{centering}
%L, B, R, T
\resizebox{0.8\columnwidth}{!}{\includegraphics[trim=0cm 0cm 0cm 0, clip=true]{D:/OwnCloudData/02_Results/LVR_50um_6mm/{Average_25.0_40.2_PulseShape}.pdf}}
\caption{The average pulse shape of the 1photoelectron in blue and the 2photoelectron pulse in red of HPK LVR 6mm at 25$^{\circ}$~C and at point of operation. Both pulses have a FWHM of around 7ns and an undershoot of 20\%, with no ringing. }
\label{fig:LVR6_PS}
\end{centering}
\end{figure}


%\newpage
\subsubsection{Gain}
\label{subsubsec:LVR6Gain}
Figure(\ref{fig:LVR6_Gain})(right) shows the gain of the LVR 6mm device. It is, as expected, linear over a long range and nearly independent of temperature when parametrized with over-voltage. The flattening of the slope to a plateau shape in the lower over-voltage range, is caused by the analysis being unable to identify peaks lower than the set threshold. Only taking into account the linear region, limits the range, where the results are relevant to an over-voltage range of $\sim$2.5V. Saturation of the oscilloscope in this range is not visible, but a check with a more expanded range revealed, that the point of saturation of the oscilloscope is at an over-voltage of $\sim$5V. The apparent overlap of the gain, when plotted against over-voltage, is based on the calculation of the breakdown-voltage being very reliable due to the large linear range. Plotted versus bias-voltage fig(\ref{fig:LVR6_Gain})(left) the expected behaviour of the gain, lowering with increasing temperature, is visible. 
\begin{figure}[h]
\begin{centering}
%L, B, R, T
\resizebox{0.4\columnwidth}{!}{\includegraphics[trim=0cm 0cm 0cm 0, clip=true]{D:/OwnCloudData/02_Results/LVR_50um_6mm/Paper/LVR_6mmGain_vs_Vb_Combined}}
\resizebox{0.39\columnwidth}{!}{\includegraphics[trim=0cm 0cm 0cm 0, clip=true]{D:/OwnCloudData/02_Results/LVR_50um_6mm/Paper/LVR_6mmGain_vs_OV_Combined}}
\caption{Gain of the HPK LVR 6mm pixel}
\label{fig:LVR6_Gain}
\end{centering}
\end{figure}


\subsubsection{Dark Count Rate and Optical Cross Talk}
\label{subsubsec:LVR6DCROCT}
\begin{figure}[h]
\begin{centering}
%L, B, R, T
\resizebox{0.45\columnwidth}{!}{\includegraphics[trim=0cm 0cm 0cm 0, clip=true]{D:/OwnCloudData/02_Results/LVR_50um_6mm/Paper/LVR_6mmDCR_vs_Vb_Combined}}
\resizebox{0.41\columnwidth}{!}{\includegraphics[trim=0cm 0cm 0cm 0, clip=true]{D:/OwnCloudData/02_Results/LVR_50um_6mm/Paper/LVR_6mmOCT_vs_OV_Combined}}
\caption{Dark Count Rate and Optical Cross Talk of the HPK LVR 6mm pixel}
\label{fig:LVR6_DCROCT}
\end{centering}
\end{figure}
The Dark Count Rate fig(\ref{fig:LVR6_DCROCT})(left), taking into account only the relevant over-voltage range of $>\sim$2.5V seems to correlate, while the resulting Optical Cross Talk fig(\ref{fig:LVR6_DCROCT})(right) is very high compared to results from the University of Nagoya, which also cover a much wider range. Only taking into account the previoulsy established relevant over-voltage range of $>\sim$2.5V, the resulting Optical Corss Talk is a factor of 2 higher. This uncertainty is a contrast to results from previous devices, where strong correlations between different groups and measurement techniques are evident. I suspect, that the device examined by me and the device present at Nagoya University are slightly different prototypes.




\clearpage
\subsection{SensL FJ60035 6mm 35$\mu$m}
The Silicon Photomultiplier by SensL with the designation FJ-60035 is also a candidate device for use in CHEC-S. It is also a 6mm device, but with a much smaller cellsize of 35$\mu$m, using the TSV technology, so there are no wire-bonds present. This results in 22292 cells on a single pixel with a fill-factor of 75\% . It is coated with plain glass. The recommended point of operation is around 30V bias-voltage, lower even than that of the HPK LVR prototypes. The device is, by the manufacturer, pre-mounted on a printed circuit board, called a test array. This test array contains a fast output, that directly couples to the cells, and a slow output, conventionally read out via the quench resistor. For the conducted tests, I used the fast output amplified with the MiniCircuits PreAMP supplied with 12V. The SensL device was the first device measured, therefore the analysis procedure used was an older iteration compared to the procedure for the Hamamatsu devices. 

\begin{figure}[h]
\begin{centering}
%L, B, R, T
\resizebox{0.45\columnwidth}{!}{\includegraphics[trim=0cm 0cm 0cm 0, clip=true]{D:/OwnCloudData/00_WriteUp/04_Thesis/Pic/SiPM_Pics/SensLPIC.JPG}}
\resizebox{0.45\columnwidth}{!}{\includegraphics[trim=0cm 0cm 0cm 0, clip=true]{D:/OwnCloudData/02_Results/SensL/SensL/Fig/{SensL_T25.0_Vb29.0.trcPulseShape}.pdf}}
\caption{SensL Test Array and pulse shape at $V_{bias-voltage} = 29V$}
\label{fig:SensL_Array_PS}
\end{centering}
\end{figure}


%\newpage
\subsubsection{Gain}
\label{subsubsec:SensLGain}

Evaluating the SensL pulse area spectra shows the 1 p.e. position and $\Delta$p.e. overlapping, so using the older analysis iteration introduced no error when evaluating the gain of the SensL device. It is clearly linear over a wide temperature range from -5$^\circ$C to 35$^\circ$C over an over-voltage ranging from 2V up to 8V. When plotted versus over-voltage the spread of the gain is even tighter signaling slight temperature independency, but still following the expected behaviour of decreasing with increasing temperature.

\begin{figure}[h]
\begin{centering}
%L, B, R, T
\resizebox{0.45\columnwidth}{!}{\includegraphics[trim=0cm 0cm 0cm 0, clip=true]{D:/OwnCloudData/02_Results/SensL/SensL/RelGain_vs_Vb_Clean}}
\resizebox{0.45\columnwidth}{!}{\includegraphics[trim=0cm 0cm 0cm 0, clip=true]{D:/OwnCloudData/02_Results/SensL/SensL/RelGain_vs_OV}}
\caption{Gain of the SensL FJ-60035 test array}
\label{fig:SensL_Gain}
\end{centering}
\end{figure}


\subsubsection{Dark Count Rate and Optical Cross Talk}
\label{subsubsec:SensL_DCROCT}
The Dark Count Rate fig(\ref{fig:SensL_DCROCT})(left) also shows the expected behaviour. At very low temperatures the changes in rate over the over-voltage range is minimal. Increasing the temperature shows a rapid increase in thermally induced dark counts. The Optical Cross Talk fig(\ref{fig:SensL_DCROCT})(right) on the other hand is independent of the device temperature, also as expected. 


\begin{figure}[h]
\begin{centering}
%L, B, R, T
\begin{overpic}[width=0.48\columnwidth,trim=0cm 0cm 0cm 0, clip=true,tics=10]{D:/OwnCloudData/02_Results/SensL/SensL/DCR_vs_OV}
\put(0,35) {Datasheet values}\put(33,13) {X}\put(58,17) {X}
\end{overpic}
\begin{overpic}[width=0.44\columnwidth,trim=0cm 0cm 0cm 0, clip=true,tics=10]{D:/OwnCloudData/02_Results/SensL/SensL/OCT_vs_OV}
\put(0,35) {Datasheet values}\put(24,13) {X}\put(51,26) {X}
\end{overpic}
%\resizebox{0.45\columnwidth}{!}{\includegraphics[trim=0cm 0cm 0cm 0, clip=true]{D:/OwnCloudData/02_Results/SensL/SensL/DCR_vs_OV}}
%\resizebox{0.45\columnwidth}{!}{\includegraphics[trim=0cm 0cm 0cm 0, clip=true]{D:/OwnCloudData/02_Results/SensL/SensL/OCT_vs_OV}}
\caption{Dark Count Rate and Optical Cross Talk of the SensL FJ-60035 test array. Datasheet values measured at 21$^\circ$C($\sim$dark red). Adding nagoya results later}
\label{fig:SensL_DCROCT}
\end{centering}
\end{figure}






%______________________________________________________________________________________________________________________________________________________________________________________________________________

\clearpage
\section{\Large Comparison}
\label{sec:comparison}
A comparison of the performance of all devices is the significant step for choosing the Silicon Photomultiplier later to be used in CHEC-S. In order to do this, all measured characteristics are compared versus over-voltage. Operation of the CHEC-S camera in GCT will come down to a decision between two operational points. The first point will be marked by an Optical Cross Talk of under 15\%. Every other attribute of the Silicon Photomultiplier at this over-voltage is then compared. This point will trade off precision for efficiency, a lower Optical Cross Talk makes real event detection easier, on the other hand, a lower Photon Detection Efficiency may forfeit a lot of potential data.\\
The second point of operation is marked at the highest achievable Photon Detection Efficiency. My conducted measurements do not involve this, other groups are comissioned to determine the point of highest Photon Detection Efficiency, in other conducted measurements, that are comparable \ref{fig:Nagoya_PDE}. This point will assure the highest detection of event photons, but will trade that for an increase in detector noise, due to the higher Dark Count Rate and more importantly Optical Cross Talk.\\
Comparing results to other groups is shown in Figure (\ref{fig:DC_DCR}), using different experimental setups and procedures and therefore also entirely different analysis techniques. Comparing against results from the University of Leicester, the University of Nagoya and the University of Catania, all of which are conducting fixed window readout of the SiPM after an expected light-pulse from a flasher-LED or pulsed laser. 

\subsection{Dark Count Rate}
\label{subsec:DC_DCR}

Comparing the Dark Count Rate of the measured devices and results from the other groups is shown in Figure \ref{fig:DC_DCR}. The differences in analysis procedure will only have a slight impact on the presumed Dark Count Rate, since all experiments record dark-count events over their respective acquisition time windows. On the other hand, if the readout window is sufficiently small, events originating from afterpulsing or delayed crosstalk could be missed. All groups experience the same multi-hit coincidence, meaning a light-event or dark-event coinciding with another, forming a (partial)multi p.e. event.\\
Only two of the 5 measured device have measurements results from other groups to compare, as it is not their focus. In the case of both, the LVR 6mm and the LCT5 7mm results can be discussed to some degree as matching. While the correlation is obvious for the LCT5 7mm device Figure(\ref{ref:DC_DCR})(red), where the covered measurement range in this paper exceeds the  external results, while matching and showing the same trend, the LVR 6mm results deviate. Between an over-voltage of 3V and 4V the results overlap, the trend on the other hand is obviously different. Additionally the limit on the higher range due to noise makes it impossible to compare against the full range measured by the external group, so the DCR for LVR 6mm must be labeled as not matching. 

\begin{figure}%[h]
\begin{centering}
%L, B, R, T
\resizebox{0.9\columnwidth}{!}{\includegraphics[trim=0cm 0cm 0cm 0, clip=true]{D:/OwnCloudData/02_Results/DeviceCompare_d2017-02-09/Presentation/Device_CompareDCR_vs_OV_Combined}}
\caption{Dark Count Rate comparison of measured devices at 25$^\circ$C. Description}
\label{fig:DC_DCR}
\end{centering}
\end{figure}

\subsection{Optical Cross Talk}
\label{subsec:DC_OCT}

The comparison of the Optical Cross Talk between the different groups and the results presented in this paper are dependant of the analysis and acquisition procedure. Extended trace analysis, utilized in this paper captures all aspects of the Optical Cross Talk, prompt and delayed as well as afterpulsing. The procedure of time window analysis, utilized by the groups being compared to, are, due to their limited window, either biased towards the prompt cross talk or in extreme cases, will not be able to capture delayed cross talk or time-delayed afterpulsing at all. Comparing data analysis techniques, for example, at the University of Leicester is therefore a vital step. Their approach utilizes a pulsed laser as light source and involves no cooling of the SiPM tile. The waveforms are extracted from the scope and a small time window is defined from the known time position window of the incident pulse to search for peaks, find their value and generate a histogram. To the pulse area histogram, a theoretical model of contributing factors is fitted. This theoretical model simulates characteristics, updating continuosly to find their correct values. Those values are the full set of characteristics of the device in testing, among them: gain curce, breakdown-voltage, OCT, PDE, noise, dynamic range, crosstalk probability.\\
There are a number of differences in their approach compared to the one utilized in this paper, most important is the time window size. If the window after an incident pulse is too short, data loss is a possibility, depending on the delay time of delayed cross-talk and afterpulsing assisted by traps with long lifetimes. This is a problem, especially with devices of the LCT5 generation implementing physical trenches isolating the cells and effectively reducing the prompt cross-talk, the contribution from the prompt cross-talk to the overall Optical Cross Talk is lowered. For time window analysis with short window times, missing data from delayed cross-talk and afterpulsing, because it will not be recorded yet, would lead to errors in the overall Optical Cross Talk results being lower than expected.\\



\begin{figure}[h]
\begin{centering}
%L, B, R, T
\resizebox{0.9\columnwidth}{!}{\includegraphics[trim=0cm 0cm 0cm 0, clip=true]{D:/OwnCloudData/02_Results/DeviceCompare_d2017-02-09/Presentation/Device_CompareOCT_vs_OV_Combined}}
\caption{Cross Talk comparison of measured devices at 25$^\circ$C. Description}
\label{fig:DC_OCT}
\end{centering}
\end{figure}

This is indeed the case for the S12642 tile in Figure \ref{fig:DC_OCT} (green). The light green bar below the dotted data presented in this paper shows the results form the University of Leicester indeed being lower. In purple results of the LCT5 S13360 device with physical trenches are shown. Compared to results from the University of Nagoya, there is a prominent upturn at aroung an over-voltage of $\sim$8V. This could be due to the differences in analysis technique. The University of Nagoya also employs time window analysis. LCT5 posseses lowered prompt cross-talk probability, so the contribution of delayed cross-talk to the overall cross-talk is higher than for S12642. With rising over-voltage the ratio between prompt and delayed cross talk shifts towards a higher contribution from delayed cross-talk. While at lower over-voltages ($\sim$2V) contributions are mostly equal, at high over-voltages ($\sim$8V) the contribution of delayed cross-talk is expected to be above 80\%, probably due to higher penetration depth and avalanche probalility.\\
Results of the LCT5 7mm device from both groups mostly overlap, the slight shift between them is most likely caused by a small error in the breakdown-voltage calculation, due to no cooling of the tile in experiments involing light. In addition, the slope of both results seems be mostly equal, and the extended range measurement, overlapping with the low-range results confirms that.\\
Even though results from 3 different groups mostly correlate, or are have at least partially understood differences, the Optical Cross Talk of LVR 6mm Figure \ref{fig:DC_OCT} (blue) compared to the results form the University of Nagoya do not show any correlation at all. This is concerning, because comparing S13360 (purple) to the same group showed strong correlation over a wide over-voltage range. Since there is also no datasheet present yet, this device is a prototype, I can only assume, that the device I examined is physically different than the device present at Nagoya. It may just be a difference in coating, which combined with the TSV technology could lead to the present uncorrelation. (quote confidential HPK talk)

Missing SensL

\subsection{Photon Detection Efficiency}

Since the measurement technique in this paper utilizes only dark counts and aims at giving an understanding of the Optical Cross Talk and temperature dependencys of the different SiPMs proposed, no PDE measurements are possible. The point of operation with the highest PDE is determined by a different group in Japan, at the University of Nagoya. Figure (\ref{fig:Nagoya_PDE}) shows the current results of their endavors. The usual procedure of comparing SiPMs is done on a plot of PDE vs OCT. This gives the most insight of the capabilities of the different devices compared to the two proposed operating points {chapter(\ref{sec:comparison}). Taking the PDE ascertained by the group at Nagoya and comparing it to the resulting Optical Cross Talk from this paper produces \\
Figure(not yet done). 


\begin{figure}[h]
\begin{centering}
%L, B, R, T
%\resizebox{0.4\columnwidth}{!}{\includegraphics[trim=0cm 0cm 0cm 0, clip=true]{D:/OwnCloudData/02_Results/OtherGroups/Nagoya/OVvsCTver9}}
\resizebox{0.45\columnwidth}{!}{\includegraphics[trim=0cm 0cm 0cm 0, clip=true]{D:/OwnCloudData/02_Results/OtherGroups/Nagoya/OVvsPDEver9}}
\resizebox{0.45\columnwidth}{!}{\includegraphics[trim=0cm 0cm 0cm 0, clip=true]{D:/OwnCloudData/02_Results/OtherGroups/Nagoya/PDEvsCTver9}}
\caption{Photon Detection Efficiency Comparison Plots from the University of Nagoya}
\label{fig:Nagoya_PDE}
\end{centering}
\end{figure}

Plotting PDE versus OCT is the standard way to compare the viability of different SiPMs, it gives a correlated overview of the two most significant characteristics. Comparing all devices at the two proposed points of operation produces the table in Figure(\ref{fig:DC_Table}). This table together with Figure (not yet done) of the 5 measured devices in this thesis is used to confirm results between groups and assist in the decision on the most viable SiPM for CHEC-S. The final decision will be taken by the group at the University of Nagoya, by Hiro Tajima, the Photosensor work group lead scientist? of GCT based on measurements on substantially more devices ref {prelim. nagoya table} .

\subsection{Point of Operation Comparison}
\begin{figure}[h]
\begin{centering}
%L, B, R, T
\resizebox{0.9\textwidth}{!}{\includegraphics[trim=0cm 0cm 0cm 4cm, clip=true]{D:/OwnCloudData/00_WriteUp/04_Thesis/Pic/Results/Comparison_Table}}
\caption{Comparison table of the measured devices in the style of the 2 proposed operation points for CHEC-S. First point represents minimal achievable OCT, second point represents maximum achievable PDE. PDE values taken from results of other groups. Note that the LCT5 6mm device can achieve even lower OCT value ($\sim$ 3 $\%$ at $V_{ov}=2.5$)}
\label{fig:DC_Table}
\end{centering}
\end{figure}


%______________________________________________________________________________________________________________________________________________________________________________________________________________
\clearpage
\section{\Large further analysis}
\subsection{Dark Count Rate recalibration with multi incident probability}

With increasing Dark Count Rate the probability of two dark events happening at the same time rises with increasing bias-voltage. Taking for example the Dark Coutn Rate of S13360 and taking the FWHM of the characteristic pulse we can calculate the probability for every dark rate at every bias-voltage and can extract a probability curve. 
Wouldnt that just make the DCR rise even higher since 1 2p.e. are now 2 1p.e.s' ??



%______________________________________________________________________________________________________________________________________________________________________________________________________________

\subsection{Prompt and delayed cross talk ratio}
What happens if we take the OCT of S13360 and scale it with an expected prompt to delayed crosstalk ratio. evidence only from confidential HPK talk, need citations. Could just make new plots in HD, evaluate.

%______________________________________________________________________________________________________________________________________________________________________________________________________________


\section{\Large Conclusion and Outlook}

Hmhmhm

For that purpose 5 different SiPM from two manufacturers have been examined.

%______________________________________________________________________________________________________________________________________________________________________________________________________________
\newpage
\section{\Large Glossary}
\begin{enumerate}
\item SiPM - Silicon Photomultiplier
\item IACT
\item CTA - Cherenkov Telescope Array
\item LST
\item MST
\item SST
\item GCT
\item CHEC



\item HPK - Hamamatsu Photonics K.K.
\item SensL - Sense Light
\end{enumerate}


%______________________________________________________________________________________________________________________________________________________________________________________________________________
\newpage
\section{\Large Bibliography}
\begin{thebibliography}{12}

\bibitem{TeVAstro} Jim Hinton et al. \textit{Teraelectronvolt Astronomy} Ann. Rev. Astron. Astrophys., 47:523

\bibitem{SCTele} Julien Rousselle et al. \textit{Construction of a Schwarzschild-Couder telescope as a candidate for the Cherenkov Telescope Array: status of the optical system}  arXiv:1509.01143v1 astro-ph.IM

\bibitem{CTADesign} The CTA Consortium \textit{Design Concepts for the Cherenkov Telescope Array CTA, An Advanced Facility for Ground-Based High-Energy Gamma-Ray Astronomy} ; arXiv:1008.3703v3 [astro-ph.IM] 11 Apr 2012

\bibitem{SST} Teresa Montaruli et al. \textit{The small size telescope projects for the Cherenkov Telescope Array} arXiv:1508.06472v1 [astro-ph.IM]

\bibitem{ASTRONET} \textit{The ASTRONET Infrastructure Roadmap} ISBN: 978-3-923524-63-1

\bibitem{HEUnivCTA} Jim Hinton et. al \textit{Seeing the High-Energy Universe with the Cherenkov Telescope Array} Astroparticle Physics 43 (2013) 1-356 

\bibitem{HAWC} R. L\'opez-Coto for the HAWC collaboration \textit{Very high energy gamma-ray astronomy with HAWC} arXiv:1612.09078v1 [astro-ph.IM] 29 Dec 2016

\bibitem{JMSensL} John Murphy \textit{SensL J-Series Silicon Photomultipliers for High-Performance Timing in Nuclear Medicine}

\bibitem{ANOtteSiPM} A. N. Otte et al. \textit{Characterization of three high efficient and blue sensitive Silicon photomultipliers} arXiv:1606.05186v2 [physics.ins-det] 26 Jan 2017

\bibitem{SiPMvsMAPMT}  A. Bouvier et al. \textit{Photosensor Characterization for the Cherenkov Telescope Array: Silicon Photomultiplier versus Multi-Anode Photomultiplier Tube} ; arXiv:1308.1390v1 [astro-ph.IM] 6 Aug 2013

\bibitem{ModelCTAP} J. Rosado S. Hidalgo \textit{Characterization and modeling of crosstalk and afterpulsing in Hamamatsu silicon photomultipliers.} arXiv:1509.02286v2 [physics.ins-det] 21 Oct 2015

\bibitem{Det_Astro} Robert G. Wagner et al. \textit{The Next Generation of Photo-Detectors for Particle Astrophysics} arXiv:0904.3565v1 [astro-ph.IM] 22 Apr 2009

\bibitem{HPK_SiPM} \textit{Opto-semiconductor handbook Chapter 03 Si APD, MPPC.} Hamamatsu Photonics K.K.

\bibitem{M_Stephan} Maurice Stephan. \textit{Design and Test of a Low Noise Amplifier for the Auger Radio Detector} Diploma Thesis, RWTH Aachen University, Juli 2009

\bibitem{RWTHMaster1} Benjamin Glau\ss. \textit{Optical Test Stand and SiPM characteriation studies.} Master's Thesis, RWTH Aachen University, June 2012.

\bibitem{uebercta}\url{http://astro.desy.de/gamma_astronomy/cta/index_eng.html}

\bibitem{ungCTA}\url{http://www.ung.si/en/research/laboratory-for-astroparticle-physics/projects/cta/}

\bibitem{FermiLAT}\url{http://www.ung.si/en/research/laboratory-for-astroparticle-physics/projects/fermi-lat/}

\bibitem{AsperaCTA}\url{http://212.71.251.65/aspera//index.php?option=com_content&task=blogcategory&id=111&Itemid=234}

\end{thebibliography}



%___________________________________________________________________________________

\newpage
\section{\Large Appendix}
\addtocontents{toc}{\protect\setcounter{tocdepth}{0}}
\appendix
\section{CTA}


\begin{figure}[h]
\centering
%L, B, R, T
\resizebox{0.5\columnwidth}{!}{\includegraphics[trim=0cm 0cm 0cm 0, clip=true]{D:/OwnCloudData/00_WriteUP/04_Thesis/Pic/Proposal/Fig/{CTA_array}.png}}
\resizebox{0.4\columnwidth}{!}{\includegraphics[trim=0cm 0cm 0cm 0, clip=true]{D:/OwnCloudData/00_WriteUP/04_Thesis/Pic/Proposal/Fig/{02_stereoscopic_technique}.jpg}}
\caption{A render of the finished CTA Array at the site in Namibia (left) with visible LSTs and MSTs, and the Shower Path Reconstruction technique of the stereoscopic view employed by current IACT experiments like HESS, MAGIC, VERITAS (right).}
\label{app:CTAPATH}

\end{figure}

\section{progenitor experiments of CTA}
%% FIG Gamma Ray
\begin{figure}[h]
\begin{centering}
%L, B, R, T
\resizebox{0.45\columnwidth}{!}{\includegraphics[trim=0cm 0cm 0cm 0, clip=true]{D:/OwnCloudData/00_WriteUP/04_Thesis/Pic/Proposal/Fig/{hess2}.jpg}}
\resizebox{0.45\columnwidth}{!}{\includegraphics[trim=0cm 0cm 0cm 0, clip=true]{D:/OwnCloudData/00_WriteUP/04_Thesis/Pic/Proposal/Fig/{magic_6_june}.jpg}}
\resizebox{0.45\columnwidth}{!}{\includegraphics[trim=0cm 0cm 0cm 0, clip=true]{D:/OwnCloudData/00_WriteUP/04_Thesis/Pic/Proposal/Fig/{veritas_New_Array}.jpg}}
\caption{IACT Projects: HESS in the Khomas Highland, Namibia. MAGIC at the Roque de los Muchachos Observatory on La Palma , one of the Canary Islands. VERITAS at Mount Hopkins, Arizona, USA}
\label{app:IACTProjects}
\end{centering}
\end{figure}



\clearpage
\section{Pulse Area Spectra of different integration window widths}
\label{subsec:PAS_window}

\begin{figure}[h]
\centering
%L, B, R, T
\resizebox{\columnwidth}{!}{\includegraphics[trim=2cm 2cm 2cm 0cm, clip=true]{D:/OwnCloudData/00_WriteUP/04_Thesis/Pic/AnalysisSteps/IntWinCompare/checs_55.png}}
\resizebox{\columnwidth}{!}{\includegraphics[trim=2cm 2cm 2cm 0cm, clip=true]{D:/OwnCloudData/00_WriteUP/04_Thesis/Pic/AnalysisSteps/IntWinCompare/checs_520.png}}
\label{app:PAS_window}
\phantomcaption
\end{figure}

\begin{figure}[h]
\ContinuedFloat
\begin{centering}
\resizebox{\columnwidth}{!}{\includegraphics[trim=2cm 2cm 2cm 0cm, clip=true]{D:/OwnCloudData/00_WriteUP/04_Thesis/Pic/AnalysisSteps/IntWinCompare/checs_1010.png}}
\caption{Pulse Area Spectra with window widths of 5 left 5 right (top), 5 left 20 right (middle), 10 left 10 right (bottom) bins respectively. Left of the 1p.e. peak of the top picture a part of the 0p.e. peak is visible. The middle figure shows the distortion an asymmetrical integration window causes. The bottom figure is the employed integration window to derive the pulse area histogram.}
\label{app:PAS_window}
\end{centering}
\end{figure}





\clearpage
\section{BreakdownVoltage}
\begin{figure}[h]
\begin{centering}
%L, B, R, T
\resizebox{0.325\textwidth}{!}{\includegraphics[trim=0cm 0cm 0cm 0, clip=true]{D:/OwnCloudData/02_Results/HPK_S12642/Paper/S12642Vbr_vs_T_Combined}}
\resizebox{0.32\columnwidth}{!}{\includegraphics[trim=0cm 0cm 0cm 0, clip=true]{D:/OwnCloudData/02_Results/LCT5_50um_6mm/Paper/LCT5_6mmVbr_vs_T_Combined}}
\resizebox{0.32\columnwidth}{!}{\includegraphics[trim=0cm 0cm 0cm 0, clip=true]{D:/OwnCloudData/02_Results/LCT5_75um_7mm/Paper/LCT5_7mmVbr_vs_T_Combined}}
\resizebox{0.32\columnwidth}{!}{\includegraphics[trim=0cm 0cm 0cm 0, clip=true]{D:/OwnCloudData/02_Results/LVR_50um_6mm/Paper/LVR_6mmVbr_vs_T_Combined}}
\resizebox{0.32\columnwidth}{!}{\includegraphics[trim=0cm 0cm 0cm 0cm, clip=true]{D:/OwnCloudData/02_Results/SensL/SensL/BreakDownVoltage_vs_Temp}}
\caption{Dependency of the breakdown-voltage of temperature for the 5 measured devices. For LCT5 7mm , the extended range measurement adds an extra datapoint at 25$^\circ$C HPK S12642 (CHEC-S) (top left) ; HPK LCT5 6mm (top middle) ; HPK LCT5 7mm (top right) ; HPK LCT5 LVR 6mm (bottom left) ; SensL FJ60035 (bottom right).}
\label{app:Device_Vbr}
\end{centering}
\end{figure}

\begin{figure}[h]
\begin{centering}
%L, B, R, T
\resizebox{0.9\textwidth}{!}{\includegraphics[trim=0cm 0cm 0cm 5.7cm, clip=true]{D:/OwnCloudData/00_WriteUp/04_Thesis/Pic/Results/BreakdownVoltage_Dependence}}
\caption{The extracted breakdown-voltage dependence of all measured devices, derived from two regression lines and their mean. For some devices the breakdown-voltage dependency is known through datasheet values. S12642 = 60mV/$^\circ$C, S13360 = 54mV/$^\circ$C, SensL FJ60035 =<21.5mV/$^\circ$C }
\label{app:Device_Vbr_Table}
\end{centering}
\end{figure}







\clearpage
\section{Pulse Area and Height Spectra}
\label{app:PAS_Clean}
Example pulse area spectra



\clearpage
\section{CHEC-S pixel comparison}
Comparison of results from 10 different pixels on the CHEC-S (HPK S12642-1616PA-50) array. Every pixel is analyzed with the same technique and analysis parameters.

\begin{figure}[h]
\begin{centering}
%L, B, R, T
%\resizebox{0.4\columnwidth}{!}{\includegraphics[trim=0cm 0cm 0cm 0, clip=true]{D:/OwnCloudData/02_Results/HPK_S12642/PIXELCOMPARE/CHECS-tile_pixel_comp_DCR_vs_Vb_Combined}}
\resizebox{0.8\columnwidth}{!}{\includegraphics[trim=0cm 0cm 0cm 0.75cm, clip=true]{D:/OwnCloudData/02_Results/HPK_S12642/PIXELCOMPARE/CHECS-tile_pixel_comp_Gain_vs_OV_Combined}}
\caption{Gain comparison of 10 different pixels of the CHEC-S (HPK S12642-1616PA-50) array}
\label{app:CHEC_S_multipixel_Gain}
\end{centering}
\end{figure}

\begin{figure}[h]
\begin{centering}
%L, B, R, T
%\resizebox{0.4\columnwidth}{!}{\includegraphics[trim=0cm 0cm 0cm 0, clip=true]{D:/OwnCloudData/02_Results/HPK_S12642/PIXELCOMPARE/CHECS-tile_pixel_comp_DCR_vs_Vb_Combined}}
\resizebox{0.8\columnwidth}{!}{\includegraphics[trim=0cm 0cm 0cm 0.75cm, clip=true]{D:/OwnCloudData/02_Results/HPK_S12642/PIXELCOMPARE/CHECS-tile_pixel_comp_DCR_vs_OV_Combined}}
\caption{DCR comparison of 10 different pixels of the CHEC-S (HPK S12642-1616PA-50) array}
\label{app:CHEC_S_multipixel_DCR}
\end{centering}
\end{figure}

\begin{figure}[ht]
\begin{centering}
%L, B, R, T
%\resizebox{0.4\columnwidth}{!}{\includegraphics[trim=0cm 0cm 0cm 0, clip=true]{D:/OwnCloudData/02_Results/HPK_S12642/PIXELCOMPARE/CHECS-tile_pixel_comp_DCR_vs_Vb_Combined}}
\resizebox{0.8\columnwidth}{!}{\includegraphics[trim=0cm 0cm 0cm 0.75cm, clip=true]{D:/OwnCloudData/02_Results/HPK_S12642/PIXELCOMPARE/CHECS-tile_pixel_comp_OCT_vs_OV_Combined}}
\caption{OCT comparison of 10 different pixels of the CHEC-S (HPK S12642-1616PA-50) array}
\label{app:CHEC_S_multipixel_OCT}
\end{centering}
\end{figure}


\newpage
\clearpage
\section{Additional Data Analysis Plots}


\begin{figure}[h]
\begin{centering}
%L, B, R, T
\resizebox{0.9\columnwidth}{!}{\includegraphics[width=0.5\textwidth]{D:/OwnCloudData/00_WriteUp/04_Thesis/Pic/Analysis/{HAM_T22.0_Vb68.5.trcRawDataZoom}.pdf}}
\caption{Description}
\label{fig:}
\end{centering}
\end{figure}

\begin{figure}[h]
\begin{centering}
%L, B, R, T
\resizebox{0.9\columnwidth}{!}{\includegraphics[width=0.5\textwidth]{D:/OwnCloudData/00_WriteUp/04_Thesis/Pic/Analysis/{HAM_T22.0_Vb68.5.trcPeakDetectLevelZoom}.pdf}}
\caption{Description}
\label{fig:}
\end{centering}
\end{figure}

\begin{figure}[h]
\begin{centering}
%L, B, R, T
\resizebox{0.9\columnwidth}{!}{\includegraphics[width=0.5\textwidth]{D:/OwnCloudData/00_WriteUp/04_Thesis/Pic/Analysis/{HAM_T22.0_Vb68.5.trcPulsePosCompZoom}.pdf}}
\caption{Description}
\label{fig:}
\end{centering}
\end{figure}

\begin{figure}[h]
\begin{centering}
%L, B, R, T
\resizebox{0.9\columnwidth}{!}{\includegraphics[width=0.5\textwidth]{D:/OwnCloudData/00_WriteUp/04_Thesis/Pic/Analysis/{HAM_T22.0_Vb68.5.trcIntegrationWindowZoom}.pdf}}
\caption{Description}
\label{fig:}
\end{centering}
\end{figure}

\clearpage
\newpage



\affidavit{B. Gebhardt}


\end{document}





%[1  ,  1.5  ,  2  ,  2.5  ,  3  ,  3.5  ,  4  ,  4.5  ,  5  ] OV
%[                           ] my OCT
%[                           ] Nag PDE

%plot in python 

