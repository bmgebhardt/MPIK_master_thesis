% !!!!!!!!!!!!!!!!!!!!!!!!!!!!!!!!!
% before compiling run:
% export TEXINPUTS=/home/gebhardt/ownCloud/00_WriteUP/03_Thesis_Proposal/TU_Da_Layout
% !!!!!!!!!!!!!!!!!!!!!!!!!!!!!!!!!
\documentclass[article,dr=phil,type=drfinal,colorback,accentcolor=tud9c]{tudthesis}
\usepackage{ngerman}

\newcommand{\getmydate}{%
  \ifcase\month%
    \or Januar\or Februar\or M\"arz%
    \or April\or Mai\or Juni\or Juli%
    \or August\or September\or Oktober%
    \or November\or Dezember%
  \fi\ \number\year%
}

\begin{document}
  \thesistitle{Die fabelhaften Benner-Boys pr"asentieren stolz das TUD-\linebreak[1]Corporate-Design f"ur {\LaTeX}!}%
    {The Fabulous Benner Boys present lordly the {\LaTeX} TUD corporate design!}
  \author{Ben Gebhardt}
  \birthplace{Darmstadt}
  \referee{Dr. Richard White}{Prof. Jim Hinton}[Prof. Tetyana Galatyuk]
  \department{Fachbereich Physik}
  \group{Institut f"ur Kernphysik}
  \dateofexam{\today}{\today}
  \tuprints{12345}{1234}
  \makethesistitle
  \affidavit{J. Walker}

  \section{Generelle Informationen}
    Die Klasse basiert auf der \textaccent{tudreport}-Klasse von C. v. Loewenich und
    J. Werner. Alle "Anderungen dort wirken sich direkt auf die
    \textaccent{tudthesis}-Klasse aus. Genauer: die \textaccent{tudthesis}-Klasse definiert nur einige
    neue Befehle und legt die Formatierung der ersten zwei Seiten (Titelseite
    und R"uckseite des Titleblattes) fest. \textbf{Alle Vordefinierten Texte sind, wie verbindlich vorgeschrieben, in der hessischen Amtssprache
    gehalten\footnote{Deutschland hat (noch) keine Amtssprache.}.}

  \section{Verwendung der Klasse}
    Die Klasse wird verwendet, indem in der Dokumentenpr"aambel
    \textaccent{\textbackslash documentclass\{tudthesis\}}
    eingetragen wird.

  \subsection{Klassenoptionen}
    Die Klasse unterst"utzt alle Klassenoptionen der tudreport-Klasse.
    \paragraph{Neue Optionen}
    \begin{itemize}
      \item \textbf{type=<dr|drfinal|diplom|msc|pp|bsc|sta>}\\
        Hiermit wird die Art der Arbeit angegeben. Dies legt verschiedene
        Formatierungen fest.\\
        \begin{tabular}{ll}
        \textaccent{dr} &f"ur eingereichte Dissertationen\\
        \textaccent{drfinal} &f"ur genehmigte Dissertationen\\
        \textaccent{diplom} &f"ur Diplomarbeiten\\
        \textaccent{msc} &f"ur Master-Theses\\
        \textaccent{pp} &f"ur Project-Proposals\\
        \textaccent{bsc} &f"ur Bachelor-Theses\\
        \textaccent{sta} &f"ur Studienarbeiten
        \end{tabular}
      \item \textbf{dr=<rernat|ing|phil>}\\
        Hiermit wird die Art des Doktorgrads angegeben.\\
        \begin{tabular}{ll}
        \textaccent{rernat} &f"ur Dr. rer. nat.\\
        \textaccent{ing} &f"ur Dr.-Ing.\\
        \textaccent{phil} &f"ur Dr. phil.
        \end{tabular}\\
        F"ur den Fall, dass der gew"unschte Titel nicht vorhanden ist, gibt
        es den Befehl \textbf{\textbackslash drtext\{\#1\}}, wobei \#1 z.B.
        \glqq Zur Erlangung des Grades eines Doktors der
        Technischen Wissenschaften (Dr.\ rer.\ tech.)\grqq\ ist.
    \end{itemize}

  \subsection{Befehle}
    \begin{itemize}\itemsep-0.5\parsep
      \item \textbf{\textbackslash thesistitle\{\#1\}\{\#2\}}\\
        \#1: Titel der Arbeit in der Erstsprache (z.B. Deutsch)\\
        \#2: Titel der Arbeit in der zweiten Sprache (z.B. Englisch)
      \item \textbf{\textbackslash makethesistitle}\\
        Erzeugt die korrekte Titelseite
      \item \textbf{\textbackslash author\{\#1\}}\\
        \#1: Name des Autors
      \item \textbf{\textbackslash birthplace\{\#1\}}\\
        \#1: Geburtsort des Autors (bei Dr.-Arbeit verbindlich)
      \item \textbf{\textbackslash date\{\#1\}}\\
        Standard ist der aktuelle Monat und das aktuelle Jahr (z.B. \getmydate)\\
        \#1: individuelles Datum
      \item \textbf{\textbackslash referee\{\#1\}\{\#2\}[\#3]}\\
        Namen der Gutachter, (3. Gutachter optional)
      \item \textbf{\textbackslash department\{\#1\}}\\
        Fachbereich an dem die Arbeit durchgef"uhrt wurde. Standard ist
        \glqq Fachbereich Physik\grqq.
      \item \textbf{\textbackslash group\{\#1\}}\\
        Arbeitsgruppe / Institut an dem die Arbeit durchgef"uhrt wurde
      \item \textbf{\textbackslash dateofexam\{\#1\}\{\#2\}}\\
        \#1: Tag der Einreichung der Arbeit\\
        \#2: Tag der Pr"ufung / Tag des Abschlusses\\
        \textaccentcolor{Wird nur bei \textaccent{type=drfinal} verwendet.
        Ansonsten wird ein leeres Feld erzeugt, in das bei Abgabe der 
        Arbeit ein Stempel gesetzt wird.}
      \item \textbf{\textbackslash tuprints\{\#1\}\{\#2\}}\\
        \#1: \textaccent{<URN-ID>} aus \textaccent{urn:nbn:de:tuda-tuprints-<URN-ID>}\\
        \#2: \textaccent{<tuprints-ID>} aus \textaccent{http://tuprints.ulb.tu-darmstadt.de/<tuprints-ID>}\\
        Entspricht der Empfehlung auf der tuprints FAQ-Seite: \textaccent{http://tuprints.ulb.tu-darmstadt.de/faq.html\#urlreservation}
      \item \textbf{\textbackslash affidavit[\#1]\{\#2\}}\\
        \#1: Datum der Eigenst"andigkeitserkl"arung (optional)\\
        \#2: Signatur unter der Unterschrift
    \end{itemize}

\end{document}
