% !!!!!!!!!!!!!!!!!!!!!!!!!!!!!!!!!
% before compiling run:
% >>> export TEXINPUTS=/home/gebhardt/ownCloud/00_WriteUP/03_Thesis_Proposal/TU_Da_Layout
% for the missing TUDa Layout packages
% note: fonts still missing
% !!!!!!!!!!!!!!!!!!!!!!!!!!!!!!!!!
\documentclass[article,type=pp,colorback,accentcolor=tud9c]{tudthesis}
\usepackage{ngerman}
\usepackage{graphicx}
\usepackage{wrapfig}
%\usepackage{hyperref}

\newcommand{\getmydate}{%
  \ifcase\month%
    \or Januar\or Februar\or M\"arz%
    \or April\or Mai\or Juni\or Juli%
    \or August\or September\or Oktober%
    \or November\or Dezember%
  \fi\ \number\year%
}

\newcommand*\rfrac[2]{{}^{#1}\!/_{#2}}

\begin{document}
  \thesistitle{SiPM Classification of the Pre-Production GCT Camera of CTA} %
    {SiPM Klassifikation der Pre-Production GCT Kamera von CTA}
  \author{Ben Gebhardt}
  \birthplace{Heidelberg}
  \referee{Dr. Richard White (MPIK)}{Prof. Jim Hinton (MPIK)}[Prof. Tetyana Galatyuk (TU DA)]
  \department{Fachbereich Physik}
  \group{Max Planck Institut f"ur Kernphysik Heidelberg}
  \dateofexam{\today}{\today}
  \tuprints{12345}{1234}
  \makethesistitle
  \affidavit{B. Gebhardt}



\section{Physical Motivation}
% missing\\
% thermal radiation example\\
% gamma ray ''types''\\
What can we learn by observing gamma rays?\\
   %% FIG FERMILAT
\begin{wrapfigure}{R}{0.3\textwidth}
\centering
\includegraphics[width=0.3\textwidth]{./Fig/{Fermi_image01}.jpg}
\caption{\label{fig:LAT}FermiLAT}
\end{wrapfigure}
%% FIG Gamma Ray
\begin{figure}[h]
\begin{centering}
%L, B, R, T
  \resizebox{0.6\columnwidth}{!}{\includegraphics[trim=0cm 0cm 0cm 0, clip=true]{./Fig/{01_astroparticle_field}.jpg}}
\caption{Gamma Radiation}
\label{fig:GammaRad}
\end{centering}
\end{figure}

Gamma radiation cannot be generated by thermal emission of hot stelar objects, the only event  with a high enough temperature to produce thermal radiation in the range of GeV and TeV gamma radiation would be the big bang, there is and has been nothing else in the known universe. So, if thermal radiation reflects the temperature of the emitting body, what do gamma rays tell us?\\
Gamma radiation probe a non-thermal universe. In this you need other mechanisms to concentrate large amounts of energy into a single quantum. But what are those?\\
There are many diverse mechanisms of emitting gamma radiation. In short gamma rays are generated by high relativistic particles, in a first step for example: accelerated by the shockwave of a supernova explosion. Those then collide with ambient gas, interact with photons or magnetic fields, by inverse compton scattering.
One such supernova remnant and also the most famous, while also being the, to date highest in energy is the Crab Pulsar. The Neutron Star located inside the Crab Nebula is the remnant of Supernova1054 and emits gamma radiation energies up to 80TeV. The source of the gamma radiation here is the so called Pulsar Wind Nebula. It is composed of highly relativistic charged particles from the pulsar interacting with the expanding Supernova remnant via inverse compton scattering.
So just like thermal radiation reflects the temperature of the emitting object, the flux and energy spectrum of the gamma rays reflect the flux and spectrum of the high energy particles. So they can be used to trace these comsic rays and electrons in distant regions of our own galaxy or even beyond.
Even higher energy gamma rays could also be the product of decays of heavy particles, like dark matter or cosmic strings.  They therefore also provide a window to the discovery of dark matter.\\ \\
The only problem is, our atmosphere is opaque for gamma radiation, gamma ray astronomy was mainly done by satellite based instruments like FermiLAT. The Farmi Large Area Telescope Satellite, launched in June 2008.

    
   \newpage
    \subsection{Project Goal}
    The goal is classification of SiPM candidate devices for the Pre-Production GCT camera of CTA. There are many competitor manufacturers of SiPMs and given that the SiPM technology is rapidly improving, CTA needs a reliable experimental setup to test current and future SiPM devices for use in Imaging Atmospheric Cherenkov Telescopes. This project will develop this experimental setup and will test the current SiPM devices to assist in the decision of the used SiPM in the GCT Camera CHEC-S, that us currently in prototyping.
    \subsection{Cherenkov Radiation}
    % farmiLAT\\
    % particle shower\\
    % triangulation of the stereoscopic view\\
    
    %% FIG Cherenkov
	\begin{wrapfigure}{R}{0.4\textwidth}
	\centering
	\includegraphics[width=0.4\textwidth]{./Fig/{gamma}.jpg}
	\caption{\label{fig:Cherenkov}Cherenkov Light}
	\end{wrapfigure}
	
    Because  our atmosphere is opaque for gamma radiation, gamma ray astronomy was mainly done by satellite based instruments like FermiLAT.
    That said, we can still see their effects on earth, in the case a gamma ray hits an air atom in the atmoshpere, getting scattered and creating a particle shower. In this airshower, the initial and each subsequent particle emitt cherenkov light.
    Cherenkov radiation is a phenomenon caused by charged particles traveling faster than the local speed of light would allow in that medium. This light is emitted in a narrow cone with an increasing angle as the particles travels downward.
    Thus, we can image the particle cascade measured with the telescope and can reconstruct their path.
    We can use this stereoscopic view of the shower to reconstruct where our source of gamma rays is in the sky.\\

    Also able to reconstruct the energy of the original photon from the amount of light produced. This is possible because energy is conserved, so all energy of the original photon is now distributed between the particles of the shower.\\

    Sources include Our Galactic Center, Supernova remnants, Pulsar Wind nebulae,  but also extragalactic sources like active galactic nuclei or gamma ray bursts\\


    
    That is, if the original particle was indeed a gamma photon. There are other charged high energy particles we could also detect, namely  atomic nuclei or electrons, which could collide with the atmosphere producing similar events. So, to determine whether it is a hadronic or gamma shower we look at many different shower characteristics like diffusion , compactness, and the ratio of width and length.


    
\section{Imaging Atmoshperic Cherenkov Telescopes}
%% FIG Gamma Ray
\begin{figure}[h]
\begin{centering}
%L, B, R, T
  \resizebox{0.29\columnwidth}{!}{\includegraphics[trim=0cm 0cm 0cm 0, clip=true]{./Fig/{hess2}.jpg}}
   \resizebox{0.29\columnwidth}{!}{\includegraphics[trim=0cm 0cm 0cm 0, clip=true]{./Fig/{magic_6_june}.jpg}}
   \resizebox{0.29\columnwidth}{!}{\includegraphics[trim=0cm 0cm 0cm 0, clip=true]{./Fig/{veritas_New_Array}.jpg}}
\caption{IACT Projects: HESS, MAGIC, VERITAS}
\label{fig:IACTProjects}
\end{centering}
\end{figure}

Before, gamma ray astronomy was mainly done usind satelites based instruments. \\
Currentground based IACTs consisted of mostly 4 telescopes,  HESS, MAGIC, Veritas amond them.\\
MAGIC at the Roque de los Muchachos Observatory on La Palma , one of the Canary Islands.\\
HESS Khomas Highland, Namibia\\
VERITAS Mount Hopkins, Arizona, USA\\

All of those arrays consist of at most 5 telescopes spread over a wide area. So most cascades are viewed by only 2 or 3 of the telescopes.
    
    \subsection{CTA}
    
    
    \begin{figure}[h]
\begin{centering}
%L, B, R, T
  \resizebox{0.5\columnwidth}{!}{\includegraphics[trim=0cm 0cm 0cm 0, clip=true]{./Fig/{CTA_array}.png}}
  \resizebox{0.4\columnwidth}{!}{\includegraphics[trim=0cm 0cm 0cm 0, clip=true]{./Fig/{02_stereoscopic_technique}.jpg}}
\caption{CTA Array and the Path Reconstruction}
\label{fig:CTAPATH}
\end{centering}
\end{figure}

%     %% FIG CTA
% \begin{figure}[h]
% \begin{centering}
% %L, B, R, T
%   \resizebox{0.6\columnwidth}{!}{\includegraphics[trim=0cm 0cm 0cm 0, clip=true]{./Fig/{CTA_array}.png}}
% \caption{CTA}
% \label{fig:ICTA}
% \end{centering}
% \end{figure}

CTA stands for Cherenkov Telescoipe Array and is an array of many tens of telescopes distributed over a larger energy range than before. It will allow detection of gamma rays over a large area on the ground and from multiple different directions. We can use this stereoscopic view of the shower to reconstruct where our source of gamma rays is in the sky.
	\newline 	

	There are currently 2 sites planed. 	
	\begin{enumerate}
	\item A southern cta site, with an array consisting of three types of telescopes:\\
		\begin{enumerate}
		\item The low energy instruments, between 20 and 200 GeV, will consist of four 23 meter class telescopes with a moderate field of view (FoV) of the order of about 4.5 degrees.
		\item The medium energy range, from around 100 GeV to 10 TeV, will be covered by 24 telescopes of the 12 meter class with a FoV of 7 degrees.
		\item The high energy instruments, operating between a few TeV to 300 TeV, will consist of about 72 small (4 meter diameter) telescopes with a FoV ranging from 9.1 to 9.6 degrees.
		\item  (area covered by the array of telescopes: ~4 $km^2$)\\
		\end{enumerate}
	\item CTA Northern Site:
		\begin{enumerate} 
		\item 4 large-size telescopes and 15 medium-size telescopes
		\item (area covered by the array of telescopes: ~0.4$ km^2$)
		\end{enumerate}
	\end{enumerate}
    \subsection{SST}
    
    Small Scale Telescope\\
    There are currently 3 different prototypes in development for the SST variant: SST2M Astri and GCT , and a signle mirror prototype SST1M\\
     I am working on one of the prototypes for the high energy telescope called SST-2M GCT utilizing a 2 mirror design called Schwarzschild-Couder.
     
     \newpage
    \subsection{GCT}
        %% FIG GCT
	\begin{wrapfigure}{R}{0.5\textwidth}
	\centering
	\includegraphics[width=0.35\textwidth]{./Fig/{Bild1}.png}
	\caption{\label{fig:SCO}Schwarzschild-Couder Optics}
	\end{wrapfigure}
    
    The 2 mirror design of the telescope allows us to utilize SiPM. Current IACTs have a parabolic optical system, which is reliable and efficient but they need to be large in order to have a large FoV due to aberrations, resulting in huge CAMERA plate scales and therefore expensive assembly of photodetectors.\\

	Schwarzschild-Couder have no such aberrations, at least not on our scale, while the IACTs can still have a very large f.o.v. up to $15\,^{\circ}$ without significant degradation of the spot size. The resulting physical pixel size is more compact than that of the single mirror optics and cost-effective	photon sensors such as multi-anode photomultipliertubes (MAPMTs) (like in CHEC-M) or silicon photomultipliers (SiPMs) can be used
	for the camera.\\
	There are of course disadvantages, the optical system will be more complex, with tighter tolerances.\\

	In summary it reduces the camera plate scale, allowing us to use SiPM photosensors.

    \subsection{CHEC}
        %% FIG CHEC
\begin{figure}[h]
\begin{centering}
%L, B, R, T
  \resizebox{0.45\columnwidth}{!}{\includegraphics[trim=0cm 0cm 0cm 0, clip=true]{./Fig/{GCTPrototype_2}.jpg}}
  \resizebox{0.45\columnwidth}{!}{\includegraphics[trim=0cm 0cm 0cm 0, clip=true]{./Fig/{GCTPrototype}.jpg}}
\caption{GCT and CHEC}
\label{fig:GCTCHEC}
\end{centering}
\end{figure}
    CHEC: compact high energy camera
	In the dual mirror telescope, applies Schwarzschild-Couder optics, resulting in a primary mirror dimension of 4m. There are currently 2 designs of CHEC being prototyped, one of them is based on SiPMs called CHEC-S.\\

	The photosensors need to follow some constraints due to temperature and high FoV of the telescope,
	reasonable performance at $20 { - } 25\,^{\circ}C$ is needed. The camera is heating up during measuring and will be cooled, while SiPMs do not need cooling.
	We need about 2048 (6x6 mm$^2$) pixels to cover the FoV , and can not go much higher in tile size due to the pitch of the plane.\\
\\  \\ 
	The other camera called CHEC-M is currently in Heidelberg and is made ready to move to Paris to be mounted on GCT.\\
\newpage
    \subsection{SiPM Characteristics}
	SiPMs are semiconductor photodetectors, that consist of many avalanche photodiodes. \\
	They are used in scientific, industrial and medical applications that require fast, high-efficient, single photon resolving	photodetectors.

	This is due to their, compared to the alternative (PMTs):
	\begin{enumerate}
	\item more mechanical robust
	\item may be exposed to ambient light (PMTs do not fare well on ambient light)
	\item low power
	\item low bias voltage
	\item no need for HV, as in PMTs
	\item being a fairly new technology it is steadily improved, meaning a new generation of SiPMs every ~5months
	\end{enumerate}

	There are 4 characteristics that will define the choice of (1) the chosen type of SiPM and Manufacturer, and (2) the operational point at either the maximum photon detection efficiency or the point where optical cross talk is lower than 10 \%. 

    \subsubsection{Gain}
    Gain defines the SiPMs characteristic of converting a photoelectron into a measurable signal. It is the amount of charge in units of electrons the device generates from one photoelectron

    \subsubsection{Dark Count Rate}
	The Dark Count Rate of an SiPM is the average rate at which the device produces a signal in the
	absence of light, originating mainly from thermally generated electrons. The expected night sky background is around ~24MHz .

    \subsubsection{Optical Cross Talk}
	In SiPMs Optical Crosstalk is a phenomenon by which a cell, fired by either an incomming or by a
	thermally generated photon, triggers an avalanche in neighboring cells. OCT can severely limit the
	photon resolution of a detector.

    \subsubsection{Photon Detection Efficiency}
	Defines the efficiency of a detector to absorb a single photon and from it, produce a measurable signal.
	It is probably the most difficult characteristic to quantify (many steps needed and cross-referenced)
	and it is influenced by several factors, like geometrical- (fill factor) and quantum-efficiency

    \subsection{SiPM Candidate Devices}
 	We plan to evaluate all current devices in development by SensL and Hamamatsu. 
 	
 	\subsection{Pulse Shape}
 	As an example the SensL SiPMs FastOutput produces a 3ns FWHM pulse with a rise time of around 300ps (\ref{fig:PulseShape}(left)).\\
In order to produce a pulse area histogram from the detected peaks, an integration window of around that value (2.4ns) is used (\ref{fig:PulseShape}(right)) . 

%% Temp Figure including sensL examples   pulse shape and zoom
\begin{figure}[ht]
\begin{centering}
%L, B, R, T
  \resizebox{0.475\columnwidth}{!}{\includegraphics[trim=0cm 0cm 0cm 0, clip=true]{./Fig/Fig_SensL/{SensL_T25.0_Vb29.0.trcPulseShape}.pdf}}
  \resizebox{0.475\columnwidth}{!}{\includegraphics[trim=0cm 0cm 0cm 0, clip=true]{./Fig/Fig_SensL/{SensL_T25.0_Vb29.0.trcIntegrationWindowZoom}.pdf}} 
\caption{Pulse Shape and integration window of the SensL SiPM | Measurements of the risetime limitied by the scope bandwidth | \textcolor{red}{measured on a single pulse}}
\label{fig:PulseShape}
\end{centering}
\end{figure}



\begin{center}
\begin{table}[h]
\begin{center}
\begin{tabular}{| p{3cm} | c | c | c |}
  \hline
  & Width & Rise & Fall \\   \hline
  Data Sheet & 3ns & 300ps & 1.5ns \\  \hline
  Measurements & 2.6ns & 1.4ns & 4.2ns\\
  \hline
 \end{tabular}
\end{center}
\caption{SensL FastOut pulse shape characteristics from Data Sheet and Measurements}
\end{table}
\end{center}

\newpage
  \section{Experimental Setup}
   % Setup Scheme Picture
   % Inside Thermal Chamber
    %Whole Setup Image

      To determine Gain, the Dark Count Rate and the Optical CrossTalk for a given device, it is cooled to the experiment temperature in a thermal chamber, that also functions as a DarkBox. After the set temperature is reached, the SiPM bias voltage is set to different values and data is acquired on the oscilloscope fig(\ref{fig:setupscheme} (left)). The SiPM is mounted on a breadboard inside the temperature chamber fig(\ref{fig:setupscheme} (right)) . BiasVoltage for both, the SiPM and the pre-amp are supplied by a remotely controlled PSU via an external throughput. Signaltransfer from the SiPM to the scope uses the same throughput. Both scope and temperature chamber use basic ehternet network communications via an external switch. The PSU is connected via serial.
    
    
                %% FIG CHEC
\begin{figure}[h]
\begin{centering}
%L, B, R, T
  \resizebox{0.5\columnwidth}{!}{\includegraphics[trim=0cm 0cm 0cm 0, clip=true]{./Fig/{Setup}.jpg}}
    \resizebox{0.4\columnwidth}{!}{\includegraphics[trim=0cm 0cm 0cm 0, clip=true]{./Fig/{Setup3}.png}}
\caption{Experimental Setup Scheme}
\label{fig:setupscheme}
\end{centering}
\end{figure}
    
    

\newpage
  \section{Signal Acquisition}
    \subsection{Data Acquisition}
                %% FIG Raw Data
	\begin{wrapfigure}{R}{0.3\textwidth}
	\centering
	\includegraphics[width=0.3\textwidth]{./Fig/{Analysis_Page/rawdata}.jpg}
	\caption{\label{fig:rawdata}Raw Data}
	\end{wrapfigure}
    Data acquisition is controlled from a single python script and begins by setting the experiment temperature on the thermal chamber (TC). Once the temperature is stable the SiPM bias voltage is set over a range of values and data is acquired on the oscilloscope. The temperature is determined to be stable automatically in the acquisition code by querying the thermal chamber until the temperature changes by less than 0.2$^{\circ}$~C over 50 seconds. At each bias voltage the oscilloscope is externally triggered to capture waveforms, each significantly longer than the typical expected pulse width of an SiPM, for example 5~$\mu$s. A unique feature of the LeCroy HDO6104 is the ability to capture multiple waveforms following a remote command, so-called segments. Typically at each bias voltage several hundred segments are recorded, resulting if 100s of $\mu$s of data. Data captured on the scope is stored on the scope SSD, and transfers to the PC whilst the TC is stabilising at the next temperature. \\ \\ 
    \subsection{Data Reduction}
    
\begin{figure}[h]
\begin{centering}
%L, B, R, T
  \resizebox{0.4\columnwidth}{!}{\includegraphics[trim=0cm 0cm 0cm 0, clip=true]{./Fig/{Analysis_Page/removepeaks}.jpg}}
  \resizebox{0.4\columnwidth}{!}{\includegraphics[trim=0cm 0cm 0cm 0, clip=true]{./Fig/{Analysis_Page/filter1}.jpg}}\\
  \resizebox{0.4\columnwidth}{!}{\includegraphics[trim=0cm 0cm 0cm 0, clip=true]{./Fig/{Analysis_Page/filter2}.jpg}}
  \resizebox{0.4\columnwidth}{!}{\includegraphics[trim=0cm 0cm 0cm 0, clip=true]{./Fig/{Analysis_Page/peakpos}.jpg}} 
\caption{Data Analysis Progress}
\label{fig:removepeaks}
\end{centering}
\end{figure}
    
With the raw data fig(\ref{fig:rawdata}), peak detection is difficult, due to underlying low and high frequency noise. 
In order to eliminate the low frequency noise the first step is to strip the waveform of all peaks, by cutting at the mean of the full waveform fig(\ref{fig:removepeaks} (top left)). Going back to the original waveform, we use the RMS of this negative part of the waveform to cut the original waveform again.\\ \\
This method proofed reliable at stripping the waveform of its peaks to isolate the noise. This so called slow-noise is then smoothed with a gaussian and subtracted from the raw data to generate 'Filtered Signal 1'.  fig(\ref{fig:removepeaks} (top right)) This also assures the waveform is always pedestal-subrtacted. This is neccesary, so random jitter in the pedestal of one segment, does not corrupt the data.\\ \\
This first 'Filtered Signal 1' is then again smoothed, this time with a much narrower window to eliminate the high frequency noise. The result is 'Filtered Signal 2' fig(\ref{fig:removepeaks} (bottom left)). Which is more suited for the detection of peaks. Important note: While peak detection is done with Filtered Signal 2, calculating PulseHeight and PulseArea is done with the resulting peak-positions fig(\ref{fig:removepeaks} (bottom right)) in Filtered Signal 1.\\ \\
	


\begin{figure}[h]
\begin{centering}
%L, B, R, T
  \resizebox{0.4\columnwidth}{!}{\includegraphics[trim=0cm 0cm 0cm 0, clip=true]{./Fig/{Analysis_Page/peakdetect}.jpg}}
  \resizebox{0.4\columnwidth}{!}{\includegraphics[trim=0cm 0cm 0cm 0, clip=true]{./Fig/{Analysis_Page/PeakPosIntWindow2}.jpg}}
\caption{Peak Detection}
\label{fig:peakdetect}
\end{centering}
\end{figure}
With the peak-Position known, we determine the PulseHeights by simple counting, and the PulseArea by integrating the bins ~3ns before and ~4ns after the peak of the 11ns FWHM Pulse fig(\ref{fig:peakdetect} ).

    
    \subsection{Data Fitting}
The Data is fitted with a gauss eq(\ref{equ:gauss}) in order to extract the gain, which will then be used to calculate DCR and OCT of the device. 
For our purposes it is sufficient to fit the 1pe peak of the pulse area histogram, which will be used for calculation of the gain fig(\ref{fig:PulseAreaFit} (left)). In order to do that we first search for the valleys of the histogram. These will be used to create a window fig(\ref{fig:PulseAreaFit} (right)) in which the fit function will operate. In the event, that the evaluation of this particular device has allready been done, the script has calculated a regression line of the gain for every T and Vb. If this is the case we skip the search for the 1pe peak in favor of a calculated peak position from the regression line.   

\begin{equation}
\begin{centering}
  a*exp(-\frac{x-x_0^2}{2\sigma^2})
\label{equ:gauss}
\end{centering}
\end{equation}


\newpage

%% Figure Pulse Area Fit
\begin{figure}[h]
\begin{centering}
%L, B, R, T
\resizebox{0.4\columnwidth}{!}{\includegraphics[page=1,trim=0cm 0cm 0cm 0cm, clip={0 0 0 0},clip]{./Fig/Analysis_page/{HAM_T22.0_Vb68.0.trcPulseArea}.pdf}}
\resizebox{0.4\columnwidth}{!}{\includegraphics[page=1,trim=0cm 0cm 0cm 0cm, clip={0 0 0 0},clip]{./Fig/Analysis_page/{HAM_T22.0_Vb68.0.trcPulseAreaFit}.pdf}}
\caption{PulseAreaFit | different stages of the analysis process}
\label{fig:PulseAreaFit}
\end{centering}
\end{figure} 



The fitting is done via pythons scipy-curve-fit. This fit is cross-referenced against the position of zero (also calculated) to extract the gain.\\
In the absence of light SiPMs produce a thermally induced noise, refered to as the Dark Count Rate (DCR), and defined here as the integral of the pulse area spectrum above 0.5 p.e. divided by the total data acqusition time. Calculation of the OCT is done via the fraction $OCT = N_{>1.5}/N_{>0.5}$.    $DCR = N_{>0.5}/T_{exp}$.  

\section{Future Measurements}

Future measurements, that are not yet covered by his work, but will in part be done at MPIK include measurements of the SiPMs angular response, photo detection efficiency and afterpulsing rate. The latter is implicity included in the DCR, and while determining the PDE will soon be done by me at MPIK, measurements of the angular response will likely be done by collaborations.

\section{Operation at GCT}
We expect the camera to operate between 25 and 35 $^\circ$C.
The estimated Night Sky Background is 24MHz in dark conditions assuming 40\% PDE. So the desired operational points are further restricted to have a DCR below the Night Sky Backgound. 
The Operation of the SiPM at GCT comes down to 2 options, either an OCT of less than 10\% or the point of maximum PDE.

\newpage
\chapter{Appendices}
\begin{enumerate}
\item Jim Hinton et al. Teraelectronvolt Astronomy Ann. Rev. Astron. Astrophys., 47:523
\item Julien Rousselle et al. Construction of a Schwarzschild-Couder telescope as a candidate for the Cherenkov Telescope Array: status of the optical system
\item CTA Consortium et al. Design Concepts for the Cherenkov Telescope Array
\item Teresa Montaruli et al. The small size telescope projects for the Cherenkov Telescope Array
\item The ASTRONET Infrastructure Roadmap ISBN: 978-3-923524-63-1
\item Jim Hinton et. al Seeing the High-Energy Universe with the Cherenkov Telescope Array Astroparticle Physics 43 (2013) 1-356 
\item John Murphy SensL J-Series Silicon Photomultipliers for High-Performance Timing in Nuclear Medicine 
\item A. N. Otte et al. Characterization of three high efficient and blue sensitive Silicon photomultipliers
\item \begin{verbatim}http://astro.desy.de/gamma_astronomie/cta/medien/ueber_cta/index_ger.html\end{verbatim}
\item \begin{verbatim}http://www.ung.si/en/research/laboratory-for-astroparticle-physics/projects/cta/\end{verbatim}
\end{enumerate}




  % \section{Results}
  %   Probably not needed in the proposal.
  %   \subsection{SensL}
  %   \subsubsection{Gain}
  %   \subsubsection{Dark Count Rate}
  %   \subsubsection{Optical Cross Talk}
  %   \subsubsection{Photon Detection Efficiency}
  %   \subsection{Hamamatsu}
  %   \subsubsection{Gain}
  %   \subsubsection{Dark Count Rate}
  %   \subsubsection{Optical Cross Talk}
  %   \subsubsection{Photon Detection Efficiency}

\end{document}
